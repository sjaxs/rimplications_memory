\subsection{Librer\'ia b\'asica para sistemas de implicaciones}
Se va a intentar que este paquete de funciones pueda perdurar en el tiempo, y para ello, 
se a realizado una librer\'ia b\'asica para intentar modularizar el c\'odigo y que sea lo 
m\'as legible posible a la hora de poder comprenderlo.

En esta librer\'ia b\'asica se encuentran diferentes funciones que se usan bastante, a continuaci\'on 
se van a detallar una a una.




%%% initialize.setOfAttributes %%%
\subsubsection{Inicializar conjunto de atributos}

    \textbf{Descripci\'on}

    Esta funci\'on se utilizar\'a para inicializar un conjunto de datos, es decir, le vamos a pasar un subconjunto 
    de todos los atributos que componen a un grupo de implicaciones, adem\'as de todos los atributos. Dentro de esta 
    se va a utilizar la funci\'on encode para convertir el subconjunto de atributos en un objeto itemMatrix y de esta 
    forma poder trabajar con ellos en los diferentes algoritmos que se van a desarrollar a lo largo de este TFG.


    \textbf{C\'odigo}



    \textbf{Ejemplo}
    Para que se entienda mejor, se va a plantear un ejemplo. 

    Si tenemos el conjunto de implicaciones siguiente: 









\subsubsection{}

    \textbf{Descripci\'on}
    

    \textbf{C\'odigo}

    
    \textbf{Ejemplo}




\subsubsection{}

    \textbf{Descripci\'on}
    

    \textbf{C\'odigo}

    
    \textbf{Ejemplo}



\subsubsection{}

    \textbf{Descripci\'on}
    

    \textbf{C\'odigo}

    
    \textbf{Ejemplo}