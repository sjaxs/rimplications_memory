\subsection{Librer\'ia b\'asica para sistemas de implicaciones}
Se va a intentar que este paquete de funciones pueda perdurar en el tiempo, y para ello, 
se a realizado una librer\'ia b\'asica para intentar modularizar el c\'odigo y que sea lo 
m\'as legible posible a la hora de poder comprenderlo.

En esta librer\'ia b\'asica se encuentran diferentes funciones que se usan bastante, a continuaci\'on 
se van a detallar una a una.




%%% initialize.setOfAttributes %%%
\subsubsection{Inicializar conjunto de atributos}

    \textbf{Descripci\'on}

    Esta funci\'on se utilizar\'a para inicializar un conjunto de datos, es decir, le vamos a pasar un subconjunto 
    de todos los atributos que componen a un grupo de implicaciones, adem\'as de todos los atributos. Dentro de esta 
    se va a utilizar la funci\'on encode para convertir el subconjunto de atributos en un objeto itemMatrix y de esta 
    forma poder trabajar con ellos en los diferentes algoritmos que se van a desarrollar a lo largo de este TFG.


    \textbf{C\'odigo}



    \textbf{Ejemplo}
    Para que se entienda mejor, se va a plantear un ejemplo. 

    Si tenemos el conjunto de implicaciones siguiente: 





%%% set.of.attributes %%%
\subsubsection{Conjunto de atributos}
    ¿Esto se ha usado alguna vez?
    \textbf{Descripci\'on}
    

    \textbf{C\'odigo}

    
    \textbf{Ejemplo}



%%% is.singleton %%%
\subsubsection{Conjunto unitario}

    \textbf{Descripci\'on}
    
    Un conjunto unitario es aquel que tiene un s\'olo elemento. Por lo que esta funci\'on 
    devolver\'a TRUE si la longitud del par\'ametro que le pasemos es igual a 1, o FALSE en caso 
    contrario.

    \'Esta es una funci\'on a la que se le puede pasar por par\'ametro una lista, un vector, 
    un conjunto de reglas o cualquier otro conjunto de elementos al que se le pueda aplicar la funci\'on 
    length().


    \textbf{C\'odigo}

    
    \textbf{Ejemplo}



%%% union.sets %%% 
\subsubsection{Uni\'on}

    \textbf{Descripci\'on}

    La uni\'on de dos conjuntos es una operaci\'on que devolver\'a otro conjunto formado por 
    todos los elementos de los dos conjuntos iniciales. Si un elemento se encuentra en los dos 
    conjuntos, s\'olo aparecer\'a una vez en el resultante.
    
    De nuevo, esta funci\'on, podr\'a ser usada por la mayor\'ia de las estructuras de datos, ya que en 
    nuestra funci\'on, lo que se usa es itemUnion, una funci\'on perteneciente al paquete arules.

    \textbf{C\'odigo}

    
    \textbf{Ejemplo}




%%% intersection.sets %%% 
\subsubsection{Intersecci\'on}

    \textbf{Descripci\'on}
    
    La intersecci\'on de dos conjuntos devolver\'a otro conjunto resultante con los elementos 
    que se encuentren en ambos conjuntos iniciales. En este caso, tambi\'en se podr\'a usar con las 
    estructuras de datos que se necesite.

    \textbf{C\'odigo}

    
    \textbf{Ejemplo}




%%% difference.sets %%% 
\subsubsection{Diferencia}

    \textbf{Descripci\'on}
    La diferencia de dos conjuntos da como resultado otro conjunto con los elementos que resultan de 
    restar a los elementos que forman el primer conjunto, los del segundo. Es decir, se obtendr\'ian 
    todos elementos del  primer conjunto que no est\'an en el segundo.

    De nuevo, en nuestra funci\'on se usa una del paquete arules para poder utilizar la gran mayor\'ia de estructuras 
    que sean compatibles con restar elementos.
    

    \textbf{C\'odigo}

    
    \textbf{Ejemplo}


%%% is.included %%% 
\subsubsection{Incluido}

    \textbf{Descripci\'on}
    Un conjunto A est\'a incluido en un conjunto B, si A es subconjunto de B. Por lo que \'esta funci\'on 
    devolver\'a TRUE si cumple dicha propiedad, o FALSE en caso contrario.

    Se puede utilizar esta funci\'on con todas aquellas estructuras de datos a las que se le pueda aplicar 
    la teor\'ia de conjuntos.

    \textbf{C\'odigo}


    \textbf{Ejemplo}


%%% is.empty %%% 
\subsubsection{Vac\'io}

    \textbf{Descripci\'on}
    Un conjunto es vac\'io si no contiene ning\'un elemento. Es decir, si su el tama\~no del conjunto es 0, 
    podemos decir que est\'a vac\'io. Esta funci\'on se podr\'a utilizar con cualquier estructura de datos 
    que se peuda medir su longitud.

    \textbf{C\'odigo}


    \textbf{Ejemplo}


    %%% equlas.sets %%% 
\subsubsection{Igualdad}

    \textbf{Descripci\'on}
    Dos conjuntos A y B son iguales si su longitud es la misma, y si para cada elemento de A, existe uno igual 
    en B y para cada elemento de B existe uno igual en A.

    \textbf{C\'odigo}


    \textbf{Ejemplo}


%%%%%%%%%%%%%%%%%%%%%%%%%%%%%%%%%%%%%%%%%%%%%%%%%%%%%%%%%%%%%%%%%%%%%%%%%%%%%%%%%%%%%%%%%%%%%%%%%%%%%%%%%%%%%%%%%%%%%%%%


%%% remove.imp %%% 
\subsubsection{Eliminar implicaci\'on}

    \textbf{Descripci\'on}
    Una de las tareas m\'as importantes de este TFG es trabajar con implicaciones. Una de las operaciones m\'as habituales 
    va a ser la de eliminar una implicaci\'on de un conjunto. Para ello, se ha definido la siguiente funci\'on, pas\'andole 
    como par\'ametros el conjunto de implicaciones y la posici\'on de la que se desea eliminar, devolver\'a un conjunto sin 
    dicha implicaci\'on.

    \textbf{C\'odigo}


    \textbf{Ejemplo}



%%% remove.all %%% 
\subsubsection{Vaciar conjunto}

    \textbf{Descripci\'on}
    Otra funci\'on que puede ser \'util es vaciar un conjunto. Se podr\'ia pensar que igualando el conjunto a nulo, 
    se podr\'ia obtener el resultado que se quiere conseguir. Pero no es as\'i, ya que lo que se quiere es un conjunto sin 
    elementos, pero que siga manteniendo su estructura de conjunto, no algo nulo. 
    
    Para ello, en esta funci\'on, vamos a devolver la posici\'on 0 del conjunto que se quiere vaciar. Recordemos, que en el 
    lenguaje R los arrays, listas y dem\'as estructuras de conjuntos comienzan en 1 y no en 0 como la gran mayor\'ia de lenguajes.
    Por lo que al devolver esta posici\'on, el resultado que se obtiene es el conjunto que se le hab\'ia pasado, pero sin ning\'un 
    elemento.

    \textbf{C\'odigo}


    \textbf{Ejemplo}



%%% add.imp.k %%% 
\subsubsection{A\~nadir implicaci\'on en posici\'on}

    \textbf{Descripci\'on}
    Igual que se puede querer eliminar una implicaci\'on, tambi\'en ser\'a posible a\~nadir una nueva en una determinada posici\'on 
    del conjunto. Para esta funci\'on necesitamos como par\'ametros, el conjunto que se quiere ampliar, la parte izquierda de la nueva 
    implicaci\'on, la derecha y la posici\'on que queremos que ocupe en el conjunto.

    Luego, en dicha funci\'on, lo primero que se hace es crear una nueva regla, con las dos partes que le hemos pasado. A continuaci\'on, 
    unimos en un mismo conjunto, todas las implicaciones que se ten\'ian en las posiciones anteriores a la nueva que se quiere insertar, 
    la nueva implicaci\'on, y las restantes que se encontraban en posiciones posteriores. Y finalmente, se devuelve este conjunto.


    \textbf{C\'odigo}


    \textbf{Ejemplo}



%%% add.imp %%% 
\subsubsection{A\~nadir implicaci\'on}

    \textbf{Descripci\'on}
    Al igual que en el caso anterior se quer\'ia a\~nadir una implicaci\'on en una determinada posici\'on, puede que d\'e igual 
    el lugar en el que se inserte. Por lo que esta funci\'on, inserta la nueva implicaci\'on en la \'ultima posici\'on del 
    conjunto. Como par\'ametros necesita los mismos que en el anterior caso, excepto la posici\'on a insertar.

    As\'i que, se crea la nueva implicaci\'on, con las dos partes que se le pasan como par\'ametros, y se le concatena al conjunto 
    inicial para devolver un conjunto ampliado con una nueva implicaci\'on.

    \textbf{C\'odigo}


    \textbf{Ejemplo}



%%% new.imp %%% 
\subsubsection{Nueva implicaci\'on}

    \textbf{Descripci\'on}


    \textbf{C\'odigo}


    \textbf{Ejemplo}


%%% read.left %%% 
\subsubsection{Leer antecedente}

    \textbf{Descripci\'on}


    \textbf{C\'odigo}


    \textbf{Ejemplo}


%%% read.right %%% 
\subsubsection{Leer consecuente}

    \textbf{Descripci\'on}


    \textbf{C\'odigo}


    \textbf{Ejemplo}


%%% substitute.imp %%% 
\subsubsection{Sustituir implicaci\'on}

    \textbf{Descripci\'on}


    \textbf{C\'odigo}


    \textbf{Ejemplo}


%%% included.left %%% 
\subsubsection{Incluido en antecedente}

    \textbf{Descripci\'on}


    \textbf{C\'odigo}


    \textbf{Ejemplo}



%%% included.left.right %%% 
\subsubsection{Antecedente incluido en consecuente}

    \textbf{Descripci\'on}


    \textbf{C\'odigo}


    \textbf{Ejemplo}


%%% delete.set.of.IS %%% 
\subsubsection{Eliminar implicaciones con consecuente vac\'io}

    \textbf{Descripci\'on}


    \textbf{C\'odigo}


    \textbf{Ejemplo}


%%% size.set %%% 
\subsubsection{Tama\~no del conjunto}

    \textbf{Descripci\'on}


    \textbf{C\'odigo}


    \textbf{Ejemplo}



%%% cadinality.set %%% 
\subsubsection{Cardinalidad del conjunto}

    \textbf{Descripci\'on}


    \textbf{C\'odigo}


    \textbf{Ejemplo}




%%% core %%% 
\subsubsection{Core}

    \textbf{Descripci\'on}


    \textbf{C\'odigo}


    \textbf{Ejemplo}




%%% body %%% 
\subsubsection{Body}

    \textbf{Descripci\'on}


    \textbf{C\'odigo}


    \textbf{Ejemplo}


