\subsection{Librer\'ia b\'asica para sistemas de implicaciones}
Se va a intentar que este paquete de funciones pueda perdurar en el tiempo, y para ello, 
se a realizado una librer\'ia b\'asica para intentar modularizar el c\'odigo y que sea lo 
m\'as legible posible a la hora de poder comprenderlo.

En esta librer\'ia b\'asica se encuentran diferentes funciones que se usan bastante, a continuaci\'on 
se van a detallar una a una.




%%% initialize.setOfAttributes %%%
\subsubsection{Inicializar conjunto de atributos}

    \textbf{Descripci\'on}

    Esta funci\'on se utilizar\'a para inicializar un conjunto de datos, es decir, le vamos a pasar un subconjunto 
    de todos los atributos que componen a un grupo de implicaciones, adem\'as de todos los atributos. Dentro de esta 
    se va a utilizar la funci\'on encode para convertir el subconjunto de atributos en un objeto itemMatrix y de esta 
    forma poder trabajar con ellos en los diferentes algoritmos que se van a desarrollar a lo largo de este TFG.


    \textbf{C\'odigo}



    \textbf{Ejemplo}
    Para que se entienda mejor, se va a plantear un ejemplo. 

    Si tenemos el conjunto de implicaciones siguiente: 





%%% set.of.attributes %%%
\subsubsection{Conjunto de atributos}
    ¿Esto se ha usado alguna vez?
    \textbf{Descripci\'on}
    

    \textbf{C\'odigo}

    
    \textbf{Ejemplo}



%%% is.singleton %%%
\subsubsection{Conjunto unitario}

    \textbf{Descripci\'on}
    
    Un conjunto unitario es aquel que tiene un s\'olo elemento. Por lo que esta funci\'on 
    devolver\'a TRUE si la longitud del par\'ametro que le pasemos es igual a 1, o FALSE en caso 
    contrario.

    \'Esta es una función a la que se le puede pasar por par\'ametro una lista, un vector, 
    un conjunto de reglas o cualquier otro conjunto de elementos al que se le pueda aplicar la funci\'on 
    length().


    \textbf{C\'odigo}

    
    \textbf{Ejemplo}



%%% union.sets %%% 
\subsubsection{Uni\'on}

    \textbf{Descripci\'on}

    La uni\'on de dos conjuntos es una operaci\'on que devolver\'a otro conjunto formado por 
    todos los elementos de los dos conjuntos iniciales. Si un elemento se encuentra en los dos 
    conjuntos, s\'olo aparecer\'a una vez en el resultante.
    
    De nuevo, esta funci\'on, podr\'a ser usada por la mayor\'ia de las estructuras de datos, ya que en 
    nuestra funci\'on, lo que se usa es itemUnion, una funci\'on perteneciente al paquete arules.

    \textbf{C\'odigo}

    
    \textbf{Ejemplo}




%%% intersection.sets %%% 
\subsubsection{Intersecci\'on}

    \textbf{Descripci\'on}
    
    La intersecci\'on de dos conjuntos devolver\'a otro conjunto resultante con los elementos 
    que se encuentren en ambos conjuntos iniciales. En este caso, tambi\'en se podr\'a usar con las 
    estructuras de datos que se necesite.

    \textbf{C\'odigo}

    
    \textbf{Ejemplo}




%%% difference.sets %%% 
\subsubsection{Diferencia}

    \textbf{Descripci\'on}
    La diferencia de dos conjuntos da como resultado otro conjunto con los elementos que resultan de 
    restar a los elementos que forman el primer conjunto, los del segundo. Es decir, se obtendr\'ian 
    todos elementos del  primer conjunto que no est\'an en el segundo.

    De nuevo, en nuestra funci\'on se usa una del paquete arules para poder utilizar la gran mayor\'ia de estructuras 
    que sean compatibles con restar elementos.
    

    \textbf{C\'odigo}

    
    \textbf{Ejemplo}


%%% is.included %%% 
\subsubsection{Incluido}

    \textbf{Descripci\'on}
    Un conjunto A est\'a incluido en un conjunto B, si A es subconjunto de B. Por lo que \'esta funci\'on 
    devolver\'a TRUE si cumple dicha propiedad, o FALSE en caso comtrario.

    \textbf{C\'odigo}


    \textbf{Ejemplo}


%%% is.empty %%% 
\subsubsection{Vac\'io}

    \textbf{Descripci\'on}


    \textbf{C\'odigo}


    \textbf{Ejemplo}


    %%% is.empty %%% 
\subsubsection{Vac\'io}

    \textbf{Descripci\'on}


    \textbf{C\'odigo}


    \textbf{Ejemplo}