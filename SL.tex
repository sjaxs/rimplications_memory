\subsubsection{Descripci\'on} 

El primer paso del desarrollo de este projecto consiste en la creaci\'on de una peque\~na librer\'ia de funciones
orientadas a aplicar la l\'ogica de simplificaciones a los conjuntos de reglas con los que se trabaja. Como 
puede ser la composici\'on de reglas, la reducci\'on, la uni\'on y la simplificaci\'on.

Estas funciones, adem\'as de formar parte de los algoritmos desarrollados, nos permiten reducir el n\'umero de 
reglas y la complejidad de las mismas, con el objetivo de minimizar los requisitos computacionales y de tiempo
necesarios para la ejecuci\'on de los algoritmos. As\'i como mantener limpio el c\'odigo de los algoritmos.

Esta l\'ogica de simplificaciones esta basada en los axiomas de Amstrong.\\

\textbf{Axioma de Reflexividad}

\begin{center}
    \(Y \subseteq X \implies X \to Y \)
\end{center}

\textbf{Axioma de Aumentatividad}

\begin{center}
    \(X \to Y \implies XZ \to YZ \ para \ cualquier \ Z \)
\end{center}

\textbf{Axioma de Transitividad}

\begin{center}
    \(X \to Y \wedge Y \to Z \implies X \to Z \)
\end{center}

Por ello, la librer\'ia contiene definida una funci\'on por cada regla de la l\'ogica de implicaciones. A continuaci\'on se detallar\'an las m\'as relevantes, como son la simplificaci\'on, la rsimplificaci\'on, la ssimplificaci\'on o la uni\'on de reglas.\\

\textbf{Uni\'on}\\
La uni\'on de reglas parte de la base de que si tenemos dos reglas (\(X \to Y , X \to Z\)) cuyos antecedentes son iguales, podemos inferir una nueva regla que tendr\'a como antecedente el m\'ismo y como consecuente la uni\'on de los consecuentes de ambas reglas, es decir:

\begin{center}
    \(X \to Y \wedge X \to Z \implies X \to YZ \)
\end{center}

\lstinputlisting{r_code/composition.R}
\bigskip
\textbf{Simplificaci\'on}\\
La simplificaci\'on de reglas se basa en que si disponemos de dos reglas \\ (\(X \to Y , U \to V\)) cuyos antecedentes no son disjuntos (\(X \subseteq U\)) y se cumple que el antecedente y el consecuente de una de ellas si son disjuntos (\(X \cap Y \neq \emptyset\)) podemos simplificar la otra regla de la siguiente forma:

\begin{center}
    \(Siendo \ X \to Y \ y \ U\to V \ dos \ reglas \ | \ X \subseteq U \wedge X \cap Y \neq \emptyset \implies (U - Y) \to (U - Y)\)
    % \(U \to V \implies (U - Y) \to (U - Y) \)
\end{center}

\lstinputlisting{r_code/simplification.R}
\bigskip
\textbf{Rsimplificaci\'on}\\
La rsimplicaci\'on es una regla derivada de la simplificaci\'on donde s\'olo se simplifica la parte derecha de la regla, es decir, su consecuente:

\begin{center}
    \(Siendo \ X \to Y \ y \ U\to V \ dos \ reglas \ | \ X \subseteq U \wedge X \cap Y \neq \emptyset \implies U \to (U - Y)\)
    % \(U \to V \implies (U - Y) \to (U - Y) \)
\end{center}

\lstinputlisting{r_code/rsimp.R}
\bigskip
% \textbf{Lsimplificaci\'on}\\

% \lstinputlisting{r_code/lsimp.R}
\textbf{Simplificaci\'on fuerte}\\

EXPLICACION?

\begin{center}
    \(Siendo \ A \to B \ y \ C\to D \ dos \ reglas \ | D \setminus (A \cup B) \neq \emptyset \wedge B \cap C \neq \emptyset\)
\end{center}

\begin{center}
    \(\implies A(C-B) \to D - (AB)\)
\end{center}

\lstinputlisting{r_code/ssimp.R}

