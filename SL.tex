\subsubsection{Descripci\'on} 
El primer paso del desarrollo del projecto consiste en la creaci\'on de una peque\!~na libreria de rutinas que seran usadas en los algoritmos a desarrollar con el fin de aplicar la logica de simplificaciones cuando sea necesario y para mantener el codigo de dichos algoritmos limpio.

Por ello, la libreria contiene definida una funci\'on por cada regla de la l\'ogica de implicaciones. A continuaci\'on se detallar\'an las mas relevantes, como son la simplificaci\'on, la rsimplificaci\'on, la ssimplificaci\'on o la uni\'on de reglas.\\
\textbf{Uni\'on}
La union de reglas parte de la base de que si tenemos dos reglas cuyos antecedentes son iguales, podemos inferir una nueva regla que tendra como antecedente el mismo y como consecuente la union de los consecuentes de ambas reglas, es decir:

\lstinputlisting{r_code/composition.R}
\subsubsection{C\'odigo} 
\lstinputlisting{r_code/sl_implications.R}
