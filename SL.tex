\subsubsection{Descripci\'on} 
El primer paso del desarrollo de este projecto consiste en la creaci\'on de una peque\~na libreria de funciones
orientadas a aplicar la l\'ogica de simplificaciones a los conjuntos de reglas con los se trabaja. Como 
puede ser la composici\'on de reglas, la reducci\'on, la uni\'on y la simplificaci\'on.

Estas funciones, adem\'as de formar parte de los algoritmos desarrollados, nos permiten reducir el n\'umero de 
reglas y la complejidad de las mismas con el objetivo de minimizar los requisitos computacionales y de tiempo
necesarios para la ejecuci\'on de los algoritmos.

Esta l\'ogica de simplificaciones esta basada en los axiomas de Amstrong.

\textbf{Axioma de Reflexividad}

\begin{center}
    \(Y \subseteq X \implies X \to Y \)
\end{center}

\textbf{Axioma de Aumentatividad}

\begin{center}
    \(X \to Y \implies XZ \to YZ \) para cualquier Z
\end{center}

\textbf{Axioma de Transitividad}

\begin{center}
    \(X \to Y \wedge Y \to Z \implies X \to Z \)
\end{center}



\subsubsection{Contexto NO} 
\newpage
\subsubsection{C\'odigo} 
\lstinputlisting{sl_implications.R}
