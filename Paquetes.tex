\section{Paquetes}
\subsection{Introducci\'on}


El lenguaje R es un lenguaje que est\'a muy fundamentado en la comunidad, dispone de una gran comunidad que 
contribuye al desarrollo, mejora y extensi\'on de este lenguaje. La unidad fundamental a la hora de compartir 
nuestro trabajo con la comunidad es el paquete. Es por ello por lo cual se va a desarrollar un estudio sobre 
c\'omo crear un paquete en el lenguaje R y c\'omo distribuirlo de forma que este\'e disponible para toda la 
comunidad y contribuir al desarrollo de este lenguaje.
Para ello primero hablaremos del equipo CRAN, los encargados de mantener y revisar todos los paquetes que
se encuentran disponibles para este lenguaje en el principal repositorio del mismo, luego hablaremos de 
c\'omo se estructura un paquete, directorios y archivos principales que ha de contener el mismo. Por \'ultimo, 
se detallar\'a todo el proceso de creaci\'on, documentaci\'on y distribuci\'on del paquete, apoyado con 
consejos, buenas pr\'acticas y recomendaciones que har\'an m\'as f\'acil este proceso y dotar\'an a nuestro 
paquete de mayor facilidad de uso, distribuci\'on, posibles ampliaciones futuras y detecci\'on y correcci\'on 
de posibles fallos.
Como herramientas fundamentales para seguir este proceso, haremos uso del entorno de desarrollo R Studio, 
de la herramienta Rtools y de los paquetes de R, devtools y \textbf{roxygen2}.

\'a
\'e
\'i
\'o
\'u
\~n

\subsection{?`Qu\'e es CRAN?}

Es el principal repositorio de paquetes estables de R, del cual se descargar\'an la mayor\'ia de los paquetes 
necesarios para un proyecto.
CRAN se mantiene gracias al esfuerzo de los voluntarios \textbf{(El equipo CRAN)} y los recursos de la fundaci\'on R. 
Este equipo se encarga de ponerse en contacto con las personas que se encargan de mantener sus propios paquetes, 
en caso de que algo no vaya bien.
Tambi\'en, en casos excepcionales, pueden modificar algo de un paquete que haya fallado al actualizar la versi\'on 
de R.
Por \'ultimo, tambi\'en encontramos al \textbf{R Core Team}, un grupo formado por unas 20 personas que se dedica a modificar 
el c\'odigo fuente de R.

\subsection{Estructura de paquetes en R}

Un paquete (\textbf{package}) es una colecci\'on de funciones, datos y c\'odigo R que se almacenan en una carpeta 
conforme a una estructura bien definida y f\'acilmente accesible para R.
Estos paquetes sirven para incrementar la potencia de R mejorando su funcionalidad base, o a\~nadiendo 
nuevas. El directorio donde se almacenan los paquetes se llama libreri\'ia (\textbf{library}).
Como ya se ha comentado, R ya tiene instalados por defecto algunos paquetes, otros muchos est\'an disponibles 
para descargarlos e instalarlos. Una vez completados los dos pasos anteriores, los paquetes tienen que ser 
cargados en cada sesi\'on que se vayan a usar.

Es importante distinguir entre tener un paquete instalado en el ordenador y tenerlo cargado en memoria:

\begin{itemize}
    \item Tenerlo instalado en el ordenador significa simplemente que en alg\'un momento lo hemos bajado de internet 
y lo hemos copiado en alg\'un directorio en que R lo puede localizar.
    \item Tenerlo cargado en memoria significa que durante nuestra sesi\'on de trabajo R ha lei\'ido el contenido del 
paquete y ha incorporado las funciones que contiene a su espacio de trabajo, de tal forma que tales funciones 
pueden ya ser invocadas y ejecutadas.
\end{itemize}

Una vez que tenemos claro que\'i es un paquete en R, veamos su estructura, esto es, directorios y archivos 
m\'as importantes que lo componen.
Un paquete en R no est\'a compuesto s\'olo del c\'odigo en dicho lenguaje, sino que tambie\'in incorpora m\'as ficheros, 
los cuales se van a detallar a continuaci\'on.
La informaci\'on b\'asica sobre un paquete se proporciona en el archivo \textbf{\textbf{DESCRIPTION}}, donde podemos ver que\'i hace 
el paquete, quie\'in es el autor o autores, quie\'in realizar\'a el mantenimiento, a que\'i versi\'on pertenece la 
documentaci\'on... entre otros datos.
En el fichero \textbf{NAMESPACE} se deber\'an especificar todos aquellos objetos que ser\'an importados o exportados del paquete.
En \textbf{LICENSE}, se incluir\'a una copia de la licencia para informar al usuario.
Para finalizar con los ficheros, podremos encontrar en algunos casos uno llamado \textbf{NEWS}, en el cual estar\'an 
incluidos los cambios que se realizan de una versi\'on a otra.
La carpeta \textbf{R} es el n\'ucleo de la estructura, aqui\'i se encuentran todos los archivos \textbf{\textbf{.R}} con el c\'odigo de las 
funciones que incluya dicho paquete.
Es recomendable no aglutinar todo el c\'odigo en un mismo archivo sino separar las funciones que tienen relaci\'on 
o que cumplen cierta funcionalidad en archivos separados, asi\'i como dar un nombre descriptivo a los archivos.
La carpeta \textbf{man} contiene los ficheros de ayuda, es decir, estos paquetes compondr\'an el “Manual de referencia” 
del paquete. Cada uno de estos ficheros, se debe corresponder a cada uno de los archivos que se encuentran 
en la carpeta R, pero en este caso, su nombre sera \textbf{nombre\_del\_archivo\textbf{.R}d}.
En la carpeta data se encontrar\'an todos los ficheros de datos que se deseen incorporar al paquete. La extensi\'on 
de estos ficheros ser\'an \textbf{\textbf{.R}Data} o \textbf{\textbf{.R}da}.
Por \'ultimo, en la carpeta \textbf{inst} estar\'an todos aquellos ficheros que deseemos que se instalen con el paquete.


En caso de tener c\'odigo en otro lenguaje, adem\'as de R, los ficheros correspondientes se incluir\'an en la carpeta 
\textbf{src}.
De estos archivos se hablar\'a m\'as adelante en la secci\'on de documentaci\'on.


\subsection{Requisitos previos}

\begin{itemize}
    \item R Studio
    \item Paquete \textbf{devtools}
    \item Paquete \textbf{\textbf{roxygen2}} 
    \item Rtools
\end{itemize}

\subsection{Crear proyecto R}

Una vez que se tiene clara la estructura de los paquetes en R y cu\'ales son los archivos
imprescindibles que deber\'a tener nuestro paquete, podemos comenzar a trabajar en su
creaci\'on. Si bien es cierto que existen diferentes formas de crear y trabajar con los paquetes
en R, en este caso lo podemos hacer tanto con R Studio, como se muestra a continuaci\'on,
como con \textbf{devtools::create(}\enquote*{ruta/del/paquete/nombreDelPaquete}\textbf{)}, ambas opciones nos
proporcionaran un \enquote*{esqueleto} del paquete con los archivos imprescindibles.
Una vez abierto R Studio, lo primero que debemos hacer es crear un nuevo proyecto para
ellos vamos a: 
\begin{center}
    \textbf{file > new project > new directory > r package} 
\end{center}

En este momento deber\'iamos ver una ventana como la siguiente:

Figura 1

En la casilla \textbf{Package name:} especificamos el nombre de nuestro proyecto; en la casilla
\textbf{Create package based on source files} podemos a\~nadir (picando en el bot\'on \textbf{Add}) los
archivos \textbf{.R} que contengan nuestras funciones, si los hemos generado previamente (no es
estrictamente necesario a\~nadirlos ahora, se pueden a\~nadir posteriormente).
Por \'ultimo, indicamos el directorio ra\'iz en el que se situar\'a el subdirectorio en que estar\'an
contenidos los archivos de nuestro proyecto. Resulta conveniente disponer de una carpeta
espec\'ifica para paquetes de R. R Studio crear\'a un subdirectorio dentro de dicha carpeta con
el nombre de nuestro nuevo paquete.
Si se quiere llevar a cabo un control de versiones del paquete se puede hacer clic en \textbf{Create
a git repository}.
Al pinchar en el bot\'on \textbf{Create Project} se crear\'a un directorio en la ubicaci\'on indicada con la
estructura necesaria para que R pueda construir el paquete:

Figura 2



\subsection{Crear/a\~nadir las funciones que componen el paquete}

A continuaci\'on, debemos a\~nadir a la carpeta R los scripts/funciones que incluye nuestro
paquete.
Para que R pueda generar autom\'aticamente el archivo de ayuda para nuestras funciones
usando \textbf{roxygen2} incluimos las siguientes l\'ineas antes del comienzo de la funci\'on:

Figura 3

Para ello debemos configurar las \textbf{Build Tools} de nuestro proyecto:

Figura 4

Una vez ajustada la configuraci\'on, pinchamos en \textbf{Install and Restart} de la pesta\~na \textbf{Build},
esto har\'a que \textbf{roxygen2} genere autom\'aticamente el archivo \textbf{.Rd} de ayuda de nuestra funci\'on
en la carpeta \textbf{man}.\\
La documentaci\'on de ayuda de nuestras funciones se detallar\'a m\'as en la secci\'on de
documentaci\'on.

\subsection{Documentaci\'on}

La documentaci\'on es una de las partes m\'as importantes del paquete en R, esta sirve para
ayudar a los usuarios a usar el paquete, para los desarrolladores que quieran extenderlo e
incluso para el propio creador del paquete, para en un futuro poder recordar para que serv\'ian
sus funciones.
Por ello, a continuaci\'on, se proceder\'a a describir los archivos m\'as importantes de
documentaci\'on que todo paquete de R debe contener.

\subsubsection{Archivo DESCRIPTION}

Uno de los archivos m\'as importantes dentro de la estructura de un paquete en R es el archivo
\textbf{DESCRIPTION}, su funci\'on es la de almacenar todos los metadatos relacionados con el
paquete al que pertenece, como pueden ser el autor o la licencia bajo la que se distribuye el
paquete.
En primer lugar, el archivo \textbf{DESCRIPTION} es lo que indica a R Studio que el directorio que lo
contiene es un paquete. Cuando creamos un proyecto e indicamos a R Studio que se va a
tratar de un paquete, este autom\'aticamente genera una plantilla del archivo \textbf{DESCRIPTION}
con los campos m\'inimamente necesarios.

Figura 5

De igual forma, si creamos el paquete haciendo uso de la herramienta \textbf{devtools}, obtendremos
una plantilla similar.

Figura 6

Este archivo usa un formato simple denominado Debian control format (DCF) cuya estructura
es muy simple, cada l\'inea consiste en un campo y un valor separados por \enquote*{:}, cuando el valor
8ocupa m\'as de una l\'inea hay que tabular las l\'ineas, como se puede ver en el campo
\textbf{DESCRIPTION} en la figura 5.
A parte de los campos de la plantilla existen otros m\'as a tener en cuenta como son Imports
y Suggests. A continuaci\'on, se proceder\'a a detallar el prop\'osito y/o contenido de cada uno
de los principales campos del archivo.
Package: no es m\'as que el nombre que recibir\'a nuestro paquete.
Type: Sirve para indicar a R Studio que se trata de un paquete (este campo no aparece al
crear nuestro paquete con devtools)
Title: Comentario en una l\'inea que resume la funcionalidad/cometido de nuestro paquete, es
usado por R Studio en la lista de paquetes disponibles, por lo que es recomendable que no
sea demasiado largo (65 caracteres m\'aximo).

Figura 7

Version: Indica la versi\'on de nuestro paquete, est\'a formada por al menos dos enteros
separados por “.” o “-” , como por ejemplo 1.0 o 0.7.2-5. R considera que la versi\'on 1.5, por
ejemplo, es igual a la versi\'on 1.5.0 y que la versi\'on 1.9.3 es menor que la versi\'on 1.10.0.
El formato de este n\'umero queda a elecci\'on del desarrollador, salvo por la restricci\'on ya
comentada de que tienen que ser al menos dos enteros separados por un punto o un guion,
aunque es importante destacar que un correcto uso de este n\'umero puede ayudar a los
usuarios que van a utilizar nuestro paquete a distinguir la magnitud de los cambios que ha
recibido.
Autor: En este caso bastar\'ia con indicar el nombre del autor o autores del paquete, aunque
el creador de devtools (Hadley Wickham) nos propone otro formato en el que se incluye el
email y el rol de la persona a la que se hace referencia el cual puede ser:

\begin{itemize}
    \item cre: el creador o mantenedor del paquete, en resumidas cuentas, la persona a la que
contactar en caso de alg\'un problema con el paquete.
    \item aut: autor, persona que ha realizado contribuciones importantes al desarrollo del
paquete.
    \item ctb: contribuidor, persona que ha realizado peque\~nas contribuciones, como parches.
    \item cph: propietario del copyright del paquete, en caso de no ser el propio autor, como
por ejemplo una compa\~n\'ia/empresa.
\end{itemize}

Un paquete debe tener al menos un autor y un mantenedor, aunque pueden ser la misma
persona.
Maintainer: Mantenedor del paquete, en caso de no usar el formato propuesto por el autor
de devtools.
Es importante que el creador o el mantenedor tengan un email de contacto, ya que este ser\'a
usado en caso de que exista alg\'un problema con el paquete y es el \'unico medio de
comunicaci\'on que usa CRAN para contactar cuando sea necesario. Adem\'as, dado que esa
direcci\'on ser\'a usada para el env\'io autom\'atico de emails por parte de CRAN, debe ser un
email perteneciente a una sola persona, no a una lista de mails.
\textbf{DESCRIPTION}: Como su propio nombre indica, es una descripci\'on m\'as detallada del cometido
de nuestro paquete, en este caso puede ocupar varias l\'ineas de hasta 80 caracteres cada
una.
License: este campo puede ser una abreviaci\'on de una licencia open source est\'andar como
puede ser GPL-2 o BSD, o un puntero a un fichero LICENSE con m\'as informaci\'on.
Si se quiere publicar el paquete en CRAN, se debe escoger una licencia est\'andar de forma
que no sea dif\'icil par CRAN saber si es legal o no distribuir nuestro paquete (ver 
https://svn\textbf{.R}-project.org/R/trunk/share/licenses/license.db)
Imports y Suggests: ambos sirven para indicar las dependencias de nuestro paquete con
respecto a otros paquetes.

Figura 8

En caso de necesitar una versi\'on espec\'ifica de un paquete lo podemos especificar entre
par\'entesis:

Figura 9

Esto resulta \'util en la forma en que, si un usuario no tiene la versi\'on requerida del paquete,
en caso de que hayamos especificado una, el usuario recibir\'a un mensaje de error indic\'andole que el 
paquete esta desactualizado, mientras que, si no especificamos qu\'e versi\'on
necesitamos, el usuario recibir\'a un error poco descriptivo y poco \'util.
Es recomendable no indicar una versi\'on exacta si no una versi\'on m\'inima necesaria.
Imports indica que un paquete es necesario para que nuestro paquete funcione, de hecho,
en el momento que el usuario instale nuestro paquete, si los paquetes declarados en Imports
no est\'an ya instalados, se instalaran autom\'aticamente. Sin embargo, no se cargar\'an (atach)
para su uso autom\'aticamente, por ello una buena pr\'actica a la hora de usar funciones de
otros paquetes es referenci\'andolas expl\'icitamente de la siguiente forma paquete::funci\'on().
Esto adem\'as mejora la lectura del c\'odigo haciendo posible identificar cu\'ales son las funciones
externas que estamos usando.
Esta forma de referenciar funciones de forma expl\'icita puede ser algo tedioso si estamos
continuamente usando funciones de otros paquetes, m\'as adelante se detallar\'an otras formas
de usar estas funciones (ver 7.4 NAMESPACE).
Suggests indica que nuestro paquete puede hacer uso de otro paquete, pero no es
estrictamente necesario, en este caso los paquetes sugeridos no son instalados de forma
autom\'atica y es el desarrollador el responsable de realizar la comprobaci\'on.

Figura 10

La forma m\'as f\'acil de a\~nadir un paquete a Imports y Suggests es mediante el uso de
devtools, otra forma es haci\'endolo a mano en el archivo \textbf{DESCRIPTION}.

Figura 11

Depends antes de la llegada de los namespaces a R, depends era la \'unica forma de que
nuestro paquete dependiese de otro, pero a partir de la llegada de los namespaces, es m\'as
recomendable usar los Imports de NAMESPACE. Tambi\'en es \'util para indicar una versi\'on
m\'inima de R requerida. Veremos esto m\'as detalladamente en el apartado dedicado al archivo
NAMESPACE.

LazyData ser\'a descrito m\'as adelante en el apartado de Datos externos.

\subsubsection{Documentaci\'on de objetos}

En este caso vamos a detallar la documentaci\'on de los distintos elementos que componen el
paquete, como pueden ser las funciones, los datasets, clases y gen\'ericos de S3, S4 y RC y
hasta el mismo paquete.
Este tipo de documentaci\'on es accesible a trav\'es de ? o help() y funciona como un
diccionario, permitiendo al usuario obtener informaci\'on sobre una funci\'on, paquete o conjunto
de datos.
Como ya se ha mencionado anteriormente, esta documentaci\'on se encuentra en la carpeta
man de nuestro proyecto en formato \textbf{.R}d. Para crear la documentaci\'on vamos a hacer uso del
paquete \textbf{roxygen2}, el cual nos har\'a m\'as f\'acil el trabajo, en lugar de crearla a mano.
Crear la documentaci\'on
Para crear esta documentaci\'on, lo primero que debemos hacer es a\~nadir comentarios
roxygen a nuestras archivos \textbf{.R}, como, por ejemplo

Figura 12

Una vez completado esto, presionamos Ctrl/Cmd + Shift + B, esto reconstruir\'a
completamente el paquete, actualizar\'a la documentaci\'on, reiniciar\'a R y recargar\'a nuestro
paquete. Ya podemos proceder a ver la documentaci\'on generada con ?. Repetiremos este
proceso hasta que la documentaci\'on quede como deseamos.
Comentarios roxygen
Estos comentarios comienzan con \#' y se colocan delante de cada funci\'on formando lo que
se llama un bloque. Estos bloques se dividen en tags (@NombreDelTag), el contenido de un
tag se extiende desde el final del nombre del tag hasta el inicio del siguiente o el final del
bloque.
Nota: dado que el car\'acter @ tiene un significado especial para roxygen, en caso de querer
introducir un @ en el texto (para un email, por ejemplo), deber\'iamos escribir @@. Lo mismo
ocurre con “%” y “\”, los cuales tienen un significado especial en Latex, para usarlos debemos
escribir “\%” y “\\” respectivamente.
Adem\'as, cada bloque incluye texto previo al primer tag, esto se denomina la introducci\'on y
se estructura de la siguiente forma:

\begin{itemize}
    \item La primera frase se corresponde con el t\'itulo de la documentaci\'on, es lo primero que
se ve cuando consultamos la ayuda de un paquete o funci\'on. Debe constar de solo
una l\'inea y acabar en punto.
    \item El segundo p\'arrafo es la descripci\'on, se muestra inmediatamente despu\'es del t\'itulo y
debe dar una breve descripci\'on de lo que hace la funci\'on.
    \item Por \'ultimo, el resto de p\'arrafos, si los hubiera, corresponden a una descripci\'on m\'as
detallada del funcionamiento de la funci\'on. Tambi\'en podemos usar el tag @section
para dividir los detalles de la funci\'on en distintas secciones.
\end{itemize}

El t\'itulo y la descripci\'on son obligatorios, mientras que los detalles son opcionales.
Nota: cada l\'inea de roxygen no debe superar los 80 caracteres y es recomendable tabular
las l\'ineas para facilitar la lectura.
Existen otros tags que facilitan la navegaci\'on entre los documentos de ayuda:

\begin{itemize}
    \item @seealso, que nos permite enlazar con otros recursos, como puede ser una p\'agina
web, otra funci\'on de nuestro paquete o incluso de otro paquete.
    \item @family, que sirve para identificar funciones que pertenecen a la misma familia
    \item @aliases, permite a\~nadir distintos “alias” a nuestro paquete/funci\'on, de forma que
sea posible encontrarlo buscando distintas palabras con “?”.
\end{itemize}

Existen m\'as tags que son situacionales, es decir, est\'an basados en el objeto el cual estamos
documentando. A continuaci\'on, vamos a estudiar estos tags en funci\'on del objeto al que
documentan.
Documentando funciones
Adem\'as del bloque introductorio y los tags ya mencionados, las funciones cuentan con una
serie de tags propios:
\begin{itemize}
    \item @param compuesto por un nombre y una descripci\'on. Sirve para describir los
par\'ametros de entrada de la funci\'on, su nombre, su tipo y para qu\'e se utilizan.
La descripci\'on debe comenzar con letra may\'uscula y acabar en un punto, puede
abarcar varias l\'ineas y no debe quedar ning\'un par\'ametro sin documentar.
Se pueden documentar al mismo tiempo par\'ametros similares y/o del mismo tipo
separando sus nombres con comas.
    \item @examples sirve para proveer c\'odigo en R que muestre un ejemplo de c\'omo funciona
la funci\'on. El c\'odigo debe funcionar sin errores. Esto es importante ya que muchos
usuarios miran primero los ejemplos.
    \item @return descripci\'on de la salida de la funci\'on, no siempre es necesario, pero es
bueno incluirlo si, por ejemplo, la funci\'on devuelve distintos tipos de salida en funci\'on
de los par\'ametros de entrada o si estamos devolviendo un objeto S3, S4 o RC.
\end{itemize}

Documentando datasets
Ver datos externos.
Documentando paquetes
Podemos usar roxygen para crear una p\'agina de ayuda propia del paquete, no asociada a
una funci\'on en particular. Esta p\'agina es accesible mediante package?nombreDelPaquete y
puede ser usada para describir los componentes m\'as importantes del paquete o las
dependencias que tiene, por ejemplo.
En este caso, dado que no se corresponde con un objeto en concreto, debemos etiquetarlo
manualmente como @docType package y @name nombreDelPaquete y poner un NULL al
final.

Figura 13

Documentaci\'on repetida
Dado que es muy probable que muchas veces tengamos que documentar funciones con los
mismos par\'ametros o incluso distintas versiones de una misma funci\'on, roxygen nos
proporciona dos mecanismos para que no tengamos que repetir de nuevo la misma
documentaci\'on, uno de ellos es la herencia de par\'ametros y el otro consiste en documentar
varias funciones en el mismo archivo.

\begin{itemize}
    \item Heredando par\'ametros: podemos hacer que una funci\'on herede la descripci\'on de
ciertos par\'ametros usando el tag @inheritParams nombreFuncion. Este tag traer\'a
toda la informaci\'on de los par\'ametros que no est\'an documentados en nuestra funci\'on,
pero s\'i en la funci\'on a la que referenciamos en nombreFuncion.
La funci\'on puede ser del mismo paquete o de otro paquete de la forma
paquete::@inheritParams nombreFuncion

Figura 14

Nota: la herencia no se encadena, esto es, la funci\'on referenciada debe definir los
par\'ametros expl\'icitamente, no podemos encadenar funciones hasta llegar a una que
los defina.
    \item Documentando varias funciones en el mismo archivo: podemos documentar
varias funciones en el mismo archivo usando @either o @describedIn. Esto debe ser
usado con precauci\'on, ya que documentar muchas funciones en un mismo archivo
puede dar lugar a documentaci\'on confusa, por ello se deben documentar funciones
con argumentos similares o funcionalidad similar o complementaria.
@describedIn genera una nueva secci\'on en la documentaci\'on con una lista donde
describe cada una de las funciones.

Figura 15

@rdname sobreescribe el nombre por defecto que genera roxygen y une
documentaci\'on de m\'ultiples funciones en un mismo archivo.
\end{itemize}

\subsubsection{Datos externos}

Hay tres maneras principales de incluir datos en un paquete, dependiendo de qu\'e queramos
hacer con ellos o qui\'en debe usarlos:

\begin{itemize}
    \item Si se quiere almacenar datos binarios y que est\'en disponibles para los usuarios, se
deben poner en el directorio data/. Este es el mejor sitio para los datasets.
    \item Si lo que se quiere es almacenar datos de an\'alisis, pero que los dem\'as usuarios no
los tengan disponibles, se deben colocar en R/sysdata\textbf{.R}da. En este caso es el mejor
sitio para los datos que necesiten nuestras funciones.
    \item Por \'ultimo, si queremos almacenar datos raw, se deben almacenar en el directorio
inst/data.
\end{itemize}

La localizaci\'on m\'as com\'un para los datos de un paquete es data/. Cada archivo de este
directorio debe ser un archivo \textbf{.R}data o \textbf{.R}da creado por la funci\'on save() y que contenga un
solo objeto, con el mismo nombre que el archivo.
Por ejemplo, si queremos a\~nadir un data.frame “datos”, se guardar\'a como “Rda” con la
instrucci\'on:
> save(datos, file=”ruta/del/paquete/miPaquete/data/datos\textbf{.R}da”)
Si \textbf{DESCRIPTION} incluye LazyData: true, entonces los datasets se cargar\'an de forma
“Lazy”. Es decir, que no ocupar\'an memoria hasta que se usen, devtools::cr\'eate() establece
LazyData a true autom\'aticamente.
Documentando datasets
Todos los objetos de data/ son siempre exportados, esto significa que deben ser
documentados.
La documentaci\'on de datos es como documentar una funci\'on con algunas diferencias. En
lugar de documentar los datos directamente, se debe documentar el nombre del dataset y
guardarlo en R/.

Figura 16

Hay dos etiquetas importantes para la documentaci\'on de datasets:

\begin{itemize}
    \item @format: proporciona una vista r\'apida del dataset. Para data frames, se debe incluir
una lista de definiciones que describan cada variable.
    \item @source: proporciona detalles sobre d\'onde est\'an los datos, generalmente una \textbackslash url{}.
\end{itemize}

Nunca se le pone @export a un data set.
Datos internos
Algunas veces, las funciones necesitan data tables pre-computados. Si los ponemos en data/,
estos datos estar\'an disponibles para todos los usuarios, lo que no es apropiado. En lugar de
eso, podemos guardarlos en R/sysdata\textbf{.R}da. Los objetos que se encuentran en este directorio
no se exportan, luego, no necesitan estar documentados. S\'olo estar\'an disponibles dentro del
paquete.
Generalmente, los datos de los paquetes deben pesar menos de un megabyte, si es m\'as
grande, se deber\'a pedir una exenci\'on. Esto es m\'as f\'acil si los datos se encuentran en el
propio paquete y no se actualizan frecuentemente.
Se debe estar seguro de que los datos est\'en comprimidos de forma \'optima.

\subsubsection{NAMESPACE}

El archivo NAMESPACE es muy importante en nuestro paquete, ya que nos va a ayudar a
encapsular nuestro paquete y que no falle por las dependencias que pueda tener de otros, ni
interfiera en otros que usen el nuestro si lo actualizamos.
Como su nombre indica, NAMESPACE proporciona “espacios” para “nombres”. Por ejemplo,
cuando importamos dos paquetes que ambos contienen una funci\'on con el mismo nombre,
podemos desambiguarla con el operador “::”. Por ejemplo, si usamos los paquetes plyr y
Hmisc, ambos tienen la funci\'on sumarize(), por lo que, dependiendo de cu\'al queramos usar,
usaremos plyr::sumarize() o Hmisc::sumarize().
NAMESPACE hace nuestro paquete autocontenido, tanto con los imports, como con los
exports. Los imports, definen c\'omo una funci\'on de un paquete encuentra una funci\'on de
otro. Los exports, nos ayudan a evitar los conflictos con otros paquetes especificando qu\'e
funciones se pueden usar fuera de nuestro paquete.

Generalmente querremos exportar un conjunto m\'inimo de funciones, ya que cuanto menos
exportemos, menos posibilidades hay de conflicto.
¿C\'omo carga R los paquetes?
Cuando se llama a una funci\'on, R tiene que encontrarla, primero busca en el entorno. Si no
lo encuentra, contin\'ua por los paquetes que hayamos adjuntado(attach) nosotros.
Hay una gran diferencia entre cargar(load) y adjuntar(attach).
Cuando cargamos un paquete, estar\'a en memoria, pero no en el path, por lo que tendremos
que usar el operador “::”. En cambio, si lo adjuntamos, s\'i que se encontrar\'a en el path. No
se puede adjuntar un paquete sin haberlo cargado antes.
Hay cuatro funciones que hacen que un paquete sea v\'alido. Difieren unas de otras,
dependiendo de si se cargan o se adjuntan y qu\'e tipo de error devuelve si no se encuentra el
paquete:

\begin{itemize}
    \item loadNamespace(“x”): carga el paquete (load) y lanza un error si no lo encuentra.
    \item library(x): adjunta el paquete (attach) y tambi\'en lanza un error si no lo encuentra.
    \item requireNamespace(“x”, quietely = TRUE): carga el paquete (load) y devuelve falso
si no lo encuentra.
    \item require(x, quietely = TRUE): adjunta el paquete (attach) y tambi\'en devuelve falso si
no lo encuentra.
\end{itemize}

Volvemos a la diferencia entre Depends e Imports en \textbf{DESCRIPTION}, ¿Cu\'ando debemos
usar cada uno?
La inclusi\'on de un paquete en cualquiera de ellos garantiza que est\'e instalado cuando sea
necesario. La principal diferencia es que Imports, carga el paquete, en cambio, Depends
tambi\'en lo adjunta.
Salvo que haya una buena raz\'on, siempre debemos enumerar los paquetes en Imports, ya
que un buen paquete debe ser autocontenido y minimizar los cambios en el entorno.
El archivo NAMESPACE
Cada l\'inea contiene una directriz: S3method(), export(), exportClasses(), adem\'as de otros.
Cada una de estas directivas describe un objeto de R, que nos indica si se exporta desde
nuestro paquete para ser usado por los dem\'as o si se importa para usarlo de forma local.
En total hay ocho directrices. Cuatro de ellas describen las exportaciones:

\begin{itemize}
    \item export(): exporta una funci\'on.
    \item exportPattern(): exporta todas las funciones que coinciden con un patr\'on.
    \item exportClasses(), exportMethods(): exporta las clases S4 y sus m\'etodos.
    \item S3method(): exporta los m\'etodos de S3.
\end{itemize}

Y las cuatro restantes describen las importaciones:
\begin{itemize}
    \item import(): importa todas las funciones de un paquete.
    \item importFrom(): importa las funciones seleccionadas.
    \item importClassesFrom(), importMethodsFrom(): importa las clases S4 y sus
m\'etodos.
    \item useDynLib(): importa una funci\'on de C.
\end{itemize}

No se recomienda escribir estas directrices a mano, sino usar \textbf{roxygen2}, ya que hace este
trabajo m\'as f\'acil y ordenado.
Generar el namespace con \textbf{roxygen2} es como generar la documentaci\'on para las funciones,
usamos los bloques y tags, a\~nadiendo los comentarios al \textbf{.R} correspondiente, ejecutamos
devtools::document() (o, Ctrl + Shift +D)y comprobamos que el archivo NAMESPACE se
haya modificado correctamente.
Ejemplo de archivo NAMESPACE generado por \textbf{roxygen2}:

Figura 17

Exports
Para que una funci\'on sea usable fuera de nuestro paquete debemos exportarla, si creamos
nuestro paquete con devtools::create(), se cre\'o un NAMESPACE temporal que exporta todo
lo que haya en nuestro paquete que no comience con un punto “.”. Si planeamos distribuir
nuestro paquete, debemos ser cuidadosos con lo que exportamos para reducir en lo posible
los conflictos con otros paquetes.
Para exportar un objeto, se debe poner “@export” en el bloque roxygen. Por ejemplo:

Figura 18

Esto generar\'a S3method(), export(), exportClasses(), o exportMethods(), dependiendo del
tipo de objeto.
Las funciones que se exporten deben estar documentadas, y se debe tener precauci\'on
cuando se cambien, ya que otras personas pueden estar us\'andolas.
¿Qu\'e debemos exportar si estamos usando S3, S4 o RC?
\renewcommand{\labelitemii}{\textendash}
\begin{itemize}

    \item S3:
Si se quiere que los dem\'as puedan crear instancias de una clase S3, debemos a\~nadir
@export al constructor de la funci\'on. Los gen\'ericos S3 son como funciones normales
de R, podemos exportarlos igual que las funciones.

Los m\'etodos S3 representan el caso m\'as complicado, ya que hay cuatro escenarios
diferentes.

\begin{itemize}
    \item   Un m\'etodo para un gen\'erico exportado: exportar todos los m\'etodos.
    \item   Un m\'etodo para un gen\'erico interno: t\'ecnicamente, no se necesita exportar estos
m\'etodos.
    \item   Un m\'etodo para un gen\'erico en un paquete requerido: se tendr\'a que importar el
gen\'erico y exportar el m\'etodo.
    \item   Un m\'etodo para un gen\'erico en un paquete recomendado: las directrices de
NAMESPACE deben referirse a funciones que est\'en disponibles.
\end{itemize}

    \item S4:
Clases: Si se quiere que los dem\'as puedan extender de nuestra clase, debemos
a\~nadir la etiqueta @export a nuestra clase, en cambio, si queremos que puedan crear
instancias para no extender, la etiqueta ir\'a en el constructor de la funci\'on, no en la
clase.
Gen\'ericos: @export si queremos que el gen\'erico sea usable p\'ublicamente.
Metodos: solo necesitamos exportar los m\'etodos para los gen\'ericos que no hayamos
definido, pero es recomendable exportar todos los m\'etodos.
    \item RC:
Igual que en S4. Debido a la forma en que RC est\'a implementado actualmente, es, en
general, imposible que sus clases se extiendan fuera de su paquete.
    \item Data:
Los archivos que est\'an en data/ no necesitan ser exportados.
\end{itemize}

Imports
NAMESPACE tambi\'en controla las funciones externas que pueden ser usadas en nuestro
paquete sin usar “::”. Esto puede confundirnos si recordamos que el archivo \textbf{DESCRIPTION}
(a trav\'es del campo Import) tambi\'en est\'a involucrado en los paquetes que importamos al
nuestro. Sin embargo, existe una diferencia muy clara, el campo Import del archivo
\textbf{DESCRIPTION} solo comprueba que el paquete est\'a instalado cuando se instala nuestro
paquete, pero no hace disponibles las funciones de dicho paquete, para ellos debemos usar
las directivas import de NAMESPACE.
Por otro lado, Depends, se encarga de que si nuestro paquete esta adjuntado (attached),
todos los paquetes que est\'en listados en Depends, tambi\'en se adjunten y, en el caso de que
nuestro paquete este cargado (loaded), estos paquetes lo est\'en tambi\'en.
Lo m\'as recomendable es que el paquete este importado en \textbf{DESCRIPTION} y no en el
NAMESPACE y referenciar a las funciones haciendo uso de “::”, esto puede suponernos m\'as
trabajo pero har\'a nuestro c\'odigo m\'as f\'acil de leer, entender y mantener. El contrario no se
cumple, todo paquete importado en NAMESPACE, debe estar tambi\'en en \textbf{DESCRIPTION}.

\begin{itemize}
    \item Funciones R:
Si solo se van a usar algunas funciones de otro paquete, lo recomendable es que el
paquete este importado en \textbf{DESCRIPTION} y referenciar a las funciones haciendo uso
de “::”.
En caso de que usemos alguna funci\'on de forma muy repetida podemos evitar el uso
de “::” importando la funci\'on con @importFrom paquete funcion.
Como ultima opci\'on, si usamos muchas funciones de forma repetida, importaremos el
paquete completo, @import paquete. Esto es poco recomendable, hace nuestro
c\'odigo dif\'icil de leer y puede generar conflictos con otros paquetes.
    \item S3:
Los gen\'ericos de S3 son funciones, as\'i que se le aplican las mismas reglas.
    \item S4:
Para usar las clases definidas en otros paquetes, debemos a\~nadir
@importClassesFrom package ClassA ClassB... junto a las clases que heredan de
las clases importadas, o junto a los m\'etodos que implementan un gen\'erico para las
clases importadas.
Para usar gen\'ericos definidos en otros paquetes, hay que a\~nadir
@importMethodsFrom package GenericA GenericB ... junto a los m\'etodos que usan
los gen\'ericos importados.
\end{itemize}

\subsection{Comprobando el paquete}

Una parte importante del desarrollo de un paquete es comprobar que el c\'odigo y la
documentaci\'on no presenten problemas, especialmente si se planea distribuir p\'ublicamente
el paquete. Para ello contamos con la herramienta R CMD check.
R CMD check es el nombre del comando que se usa desde una terminal, pero en nuestro
caso vamos a usar la herramienta que nos proporciona devtools, en concreto,
devtools::check() o, equivalentemente, Ctrl + Shift + E en R Studio.
Esta herramienta se encarga de que toda la documentaci\'on del paquete este actualizada, ya
que ejecuta devtools::document() de manera autom\'atica.
Encapsula el paquete antes de realizar las comprobaciones. Esto asegura que la
comprobaci\'on del paquete se hace desde un estado limpio ya que el encapsulamiento no
contiene ning\'un posible archivo temporal que se haya podido acumular y que pueda interferir
en las comprobaciones.
La mejor forma, aunque tediosa, de comprobar el paquete es la siguiente:

\begin{itemize}
    \item Ejecutar devtools::check() o Ctrl + Shift + E en R Studio.
    \item Arreglar el primer problema.
    \item Repetir los dos anteriores hasta que no haya m\'as problemas.
\end{itemize}

R CMD check devuelve tres tipos de problemas distintos:

\begin{itemize}
    \item Error: Problema grave que deber ser arreglado, aunque no planeemos subir el
paquete a CRAN.
    \item Warning: Problema medio que debe ser arreglado en caso de que planeemos subir
el paquete a CRAN.
    \item Note: Problema medio que deber\'iamos arreglar en caso de querer subir el paquete a
CRAN incluso si se trata de un falso positivo. En el caso de que no tengamos ninguna
nota, el proceso de subida a CRAN ser\'a autom\'atico sin intervenci\'on humana. Si no
es posible eliminar una nota, debemos describir por qu\'e no supone un problema.
\end{itemize}


\subsection{Liberando el paquete}

Si queremos que nuestro paquete tenga cierta trascendencia en la comunidad de R,
necesitamos subirlo a CRAN, ya que la mayor\'ia de usuarios descargan e instalan sus
paquetes desde CRAN.
A continuaci\'on, se detallar\'an las buenas pr\'acticas a la hora de distribuir nuestro paquete a
en CRAN.
N\'umero de versi\'on.
Ya hablamos acerca de los n\'umeros de versi\'on de nuestro paquete en la secci\'on de
documentaci\'on, en este caso, detallaremos algunos consejos a la hora de publicar una nueva
versi\'on de nuestro paquete.
El nuevo n\'umero de versi\'on debe hacer que el usuario pueda entender la magnitud de los
cambios realizados con respecto a la versi\'on anterior, por ejemplo, supongamos que la
versi\'on actual de nuestro paquete es la 0.7.3, en funci\'on de los cambios que hayamos hecho
a nuestro paquete debemos elegir un nuevo n\'umero de versi\'on u otro:

\begin{itemize}
    \item Parche: si solo hemos arreglado errores de nuestro paquete sin a\~nadir ninguna
funcionalidad significativa, un buen n\'umero de versi\'on seria 0.7.4.
    \item Actualizaci\'on menor: si adem\'as de solucionar errores hemos a\~nadido nuevas
funcionalidades y existen cambios en la compatibilidad hacia atr\'as de nuestro
paquete, el nuevo n\'umero de versi\'on deber\'ia ser 0.8.0.
    \item Actualizaci\'on mayor: si hemos realizado cambios importantes en el paquete y no
existe una compatibilidad hacia atr\'as y esto puede afectar a los usuarios, lo usual
ser\'ia un n\'umero de versi\'on como 1.0.0.
\end{itemize}

Compatibilidad hacia atr\'as.
Dado que la mayor\'ia de usuarios actualizan sus paquetes sin tener en cuenta que la nueva
versi\'on pueda dejar de ser compatible con alguna funci\'on que usan habitualmente es
importante ser cautelosos a la hora de introducir grandes cambios en la compatibilidad hacia
atr\'as de nuestro paquete.
La importancia de la compatibilidad hacia atr\'as de nuestro paquete es directamente
proporcional al n\'umero de usuarios del mismo. Por otro lado, cuanto m\'as tiempo dediquemos
al mantenimiento de esta compatibilidad m\'as dif\'icil ser\'a de a\~nadir nuevas funcionalidades,
de arreglar bugs y de mantener nuestro c\'odigo. Es por ello, que, aunque no debemos dejar
de lado esta compatibilidad, tampoco tiene que paralizarnos.
Dado que hay casos en los que es necesario romper con esta compatibilidad, lo mejor es
hacerlo gradualmente, por ello, se van a detallar algunas t\'ecnicas/consejos:

\begin{itemize}
    \item En lugar de eliminar autom\'aticamente una funci\'on, depreciarla primero y eliminarla un
par de versiones despu\'es, por ejemplo. Para ello, usamos .Deprecated().

Figura 19

Figura 20

    \item De igual forma, advertir al usuario si vamos a eliminar un argumento

Figura 21

Figura 22

    \item En el caso de que queramos depreciar mucho c\'odigo, podemos incluir una funci\'on de
ayuda que autom\'aticamente se encargue de emitir los mensajes de error o de aviso
como hace el paquete ggplot2 con su funci\'on gg\_dep.
    \item Cambios significantes en una funci\'on tienen que ser bien planificados y realizados de
forma gradual, advirtiendo el usuario, para causar el menor da\~no posible.
    \item Por \'ultimo, podemos realizar una comprobaci\'on antes de ejecutar la funci\'on, de forma
que compruebe que versi\'on del paquete tiene el usuario y as\'i ejecutar una versi\'on de
nuestra funci\'on u otra.

Figura 23
\end{itemize}

Comentarios CRAN.
Debemos recordar que CRAN est\'a compuesto por voluntarios que dedican parte de su tiempo
libre a revisar los paquetes que se suben, por tanto, cuanto m\'as f\'acil les hagamos el trabajo,
m\'as posibilidades hay de que nuestro paquete sea aceptado sin ning\'un problema.
Por ello, es recomendable a\~nadir a nuestro paquete una serie de comentarios que faciliten la
tarea de revisi\'on, estos comentarios ir\'an en un archivo llamado cran-comments.md, veamos
un ejemplo (ggplot2):

\begin{itemize}
    \item Figura 24

    \item Test environments: indica en qu\'e plataformas se ha comprobado el paquete. Dado
que R est\'a en continua evoluci\'on, es recomendable usar siempre la \'ultima versi\'on de
desarrollo del lenguaje y comprobar que nuestro paquete funciona correctamente en
al menos dos plataformas (Windows, Linux o MAC).
    \item Check results: contendr\'a una lista de errores, advertencias o notas, aunque debemos
asegurarnos de que no hay errores o advertencias (ya que, de ser as\'i, no ser\'a
aceptado en CRAN).
En el caso de que hubiera alguna nota que no hayamos podido arreglar, incluir\'iamos
el mensaje obtenido con la nota y una explicaci\'on de por qu\'e no est\'a arreglada o
porque consideramos que no afecta al correcto funcionamiento del paquete.
Es importante eliminar las notas, ya que de esta forma es m\'as probable que nuestro
paquete no requiera revisi\'on humana y el proceso de subida sea m\'as r\'apido.
    \item Reverse dependencies: sirve para indicar que si hay errores de dependencias con
otros paquetes en el caso de que estemos subiendo una nueva versi\'on de nuestro
paquete y los cambios hayan provocado errores en las dependencias con otros
paquetes. En ese caso, podemos contactar con el autor del paquete para indicar lo
que ocurre y buscar una soluci\'on.
\end{itemize}

Las pol\'iticas en CRAN.
Existen ciertas pol\'iticas que debemos cumplir a la hora de subir nuestro paquete a CRAN,
estas son comprobadas manualmente. Los voluntarios de CRAN comprueban que nuestro
paquete cumple estas pol\'iticas rigurosamente, especialmente si es la primera vez que lo
subimos.
Algunos problemas comunes son:
\begin{itemize}
    \item Ausencia del email del mantenedor del paquete. Este email es la \'unica v\'ia de contacto
que tiene CRAN en caso de alg\'un problema con nuestro paquete. En caso de que no
haya email de contacto, ante cualquier problema, el paquete ser\'a retirado de CRAN.
    \item Es importante dejar claro los propietarios del copyright y si hemos usado c\'odigo
externo, comprobar que la licencia es compatible.
    \item Paquetes que no funcionan en al menos dos plataformas no ser\'an considerados.
    \item No hacer cambios externos sin permiso del usuario: cambiar opciones, instalar
paquetes, abrir software externo, cerrar R, etc.
    \item No subir actualizaciones del paquete muy frecuentemente. La pol\'itica sugiere hacerlo
cada 1 o 2 meses como mucho.
\end{itemize}

Archivos importantes.
Una vez que nuestro paquete ya est\'a listo para subirlo, debemos tener en cuenta dos archivos
importantes m\'as, README.md, el cual proporciona una descripci\'on de nuestro paquete y su
funcionamiento y uso, y NEWS.md, el cual describe los cambios con respecto a la versi\'on
anterior de nuestro paquete. Dado que estos dos archivos no son soportados por CRAN no
hay que a\~nadirlos a la hora de construir el paquete (devtools::use\_build\_ignore("NEWS.md"),
devtools::use\_build\_ignore("README.md"))

Lanzamiento.
Lleg\'o la hora de construir nuestro paquete y subirlo a CRAN. Para construir nuestro paquete
usamos devtools::release(). Esto se encarga de volver a comprobar el paquete una \'ultima
vez y nos hace una serie de preguntas acerca de buenas pr\'acticas a modo de comprobaci\'on,
devtools::release(), autom\'aticamente, sube el paquete a CRAN.
Una vez subido, recibiremos un email de confirmaci\'on a los pocos minutos pidi\'endonos
confirmar nuestro email. Despu\'es los voluntarios de CRAN comprobar\'an el paquete y nos
devolver\'an los resultados, esto normalmente tarda desde 24h hasta 5 d\'ias.
En caso de fallo, este vendr\'a descrito en los resultados de la comprobaci\'on, en este caso,
debemos arreglar el fallo, incluir en cran-coments un comentario indicando que vamos a
hacer una resubida, actualizar las dependencias en caso de que sea necesario y ejecutar
devtools::submit\_cran() para evitar contestar de nuevo todas las preguntas de
devtools::release().
Una vez que CRAN ha aceptado nuestro paquete, construir\'a el paquete para las distintas
plataformas, esto puede dar lugar a nuevos errores que debemos arreglar en un parche y
volver a subir.