\subsubsection{Descripci\'on} 
Antes de entrar en el concepto de D-Basis, se detallar\'an otros conceptos con el fin de entrar en contexto.

Se parte de un conjunto de reglas, \( \Sigma \), y un conjunto de atributos \( M \).\\

\textbf{Operador de cierre at\'omico}\\
Se parte de que \(\Sigma\) cumple las siguientes propiedades:
\begin{itemize}
    \item \(\oslash^+_{\Sigma} = \oslash\), es decir, el cierre del vac\'io con respecto a \(\Sigma\) es el vac\'io.
    \item \(\forall x,y \in M, \{x\}^+_{\Sigma} = \{y\}^+_{\Sigma} \implies x = y\)
\end{itemize}
Se define el operador de cierre at\'omico como: \\
\begin{center}
    \((-)^*_{\Sigma}::2^M \to 2^M \) \\
    \(X^*_{\Sigma} = \cup_{\substack{x \in X}} \{x\}^+_{\Sigma} \ \forall \ X \subseteq M \)
\end{center}


\textbf{Covers}\\
Sea \(X \subseteq M\) y \(x \in X\), se dice que \(X\) es un cover de \(x\) si \(x \in X^+_{\Sigma} \setminus X\), denotado por \(x \sim X\)

\textbf{Propers covers} \\
Sea \(X \subseteq M\) y \(x \in X\), se dice que \(X\) es un proper cover de \(x\) si \(x \in X^+_{\Sigma} \setminus X^*_{\Sigma}\), denotado por \(x \dot\sim X\)

\textbf{Minimals propers covers}\\
Sea \(X \subseteq M\) y \(x \in X\), se dice que \(X\) es un minimal proper cover de \(x\) si de cumple que:
\begin{itemize}
    \item \(x \dot\sim X\)
    \item \(x \dot\sim Z\) y \(Z \subseteq X^*_{\Sigma} \implies X \subseteq Z\)
\end{itemize}

\textbf{DBasis}\\
Podemos definir la DBasis para \(\Sigma\) como el par \(<\Sigma_a, \Sigma_n>\), donde:

\begin{itemize}
    \item \(\Sigma_a = \{y \to x | y \in M, x \in X^+_{\Sigma}, x \neq y \} \)
    \item \(\Sigma_n = \{X \to x | X \subseteq M, x \in X, X \ minimal \ proper \ cover \ de \ x\} \)

\end{itemize}

\textbf{Teor\'ema}\\
Sea \(\Sigma\) un sistema implicacional. Si \(<\Sigma_a, \Sigma_n>\) es DBasis de \(\Sigma\) y \(\Sigma_d = \Sigma_a\cup \Sigma_n\) de forma que las implicaciones de \(\Sigma_a\) aparecen antes que las de \(\Sigma_n\), entonces \(\Sigma_d\) es una base ordenada directa de forma que \(\Sigma_d \equiv \Sigma\).
\textbf{Comentar el algoritmo a partire de todo esto}
\newpage 
\subsubsection{C\'odigo} 
\lstinputlisting{r_code/d.basis.R}
\newpage
\subsubsection{Ejemplo} 