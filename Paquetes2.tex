\section{Paquetes}
\subsection{Introducci\'on}


El lenguaje R es un lenguaje que est\'a muy fundamentado en la comunidad, dispone de una gran comunidad que 
contribuye al desarrollo, mejora y extensi\'on de este lenguaje. La unidad fundamental a la hora de compartir 
nuestro trabajo con la comunidad es el paquete. Es por ello por lo cual se va a desarrollar un estudio sobre 
c\'omo crear un paquete en el lenguaje R y c\'omo distribuirlo de forma que este\'e disponible para toda la 
comunidad y contribuir al desarrollo de este lenguaje.

CRAN es el principal repositorio de paquetes estables de R, del cual se descargarán la mayoría de los paquetes necesarios para un proyecto.

\'a
\'e
\'i
\'o
\'u
\~n

\subsection{Estructura de paquetes en R}

Un paquete (\textbf{package}) es una colecci\'on de funciones, datos y c\'odigo R que se almacenan en una carpeta 
conforme a una estructura bien definida y f\'acilmente accesible para R.
Estos paquetes sirven para incrementar la potencia de R mejorando su funcionalidad base, o a\~nadiendo 
nuevas.
Un paquete en R no est\'a compuesto s\'olo del c\'odigo en dicho lenguaje, sino que tambie\'in incorpora m\'as ficheros, 
los cuales se van a detallar a continuaci\'on.
La informaci\'on b\'asica sobre un paquete se proporciona en el archivo \textbf{\textbf{DESCRIPTION}}, donde podemos ver que\'i hace 
el paquete, quie\'in es el autor o autores, quie\'in realizar\'a el mantenimiento, a que\'i versi\'on pertenece la 
documentaci\'on... entre otros datos.
En el fichero \textbf{NAMESPACE} se deber\'an especificar todos aquellos objetos que ser\'an importados o exportados del paquete.
En \textbf{LICENSE}, se incluir\'a una copia de la licencia para informar al usuario.
Para finalizar con los ficheros, podremos encontrar en algunos casos uno llamado \textbf{NEWS}, en el cual estar\'an 
incluidos los cambios que se realizan de una versi\'on a otra.
La carpeta \textbf{R} es el n\'ucleo de la estructura, aqui\'i se encuentran todos los archivos \textbf{\textbf{.R}} con el c\'odigo de las funciones que incluya dicho paquete.
Es recomendable no aglutinar todo el c\'odigo en un mismo archivo sino separar las funciones que tienen relaci\'on 
o que cumplen cierta funcionalidad en archivos separados, asi\'i como dar un nombre descriptivo a los archivos.
La carpeta \textbf{man} contiene los ficheros de ayuda, es decir, estos paquetes compondr\'an el “Manual de referencia” 
del paquete. Cada uno de estos ficheros, se debe corresponder a cada uno de los archivos que se encuentran 
en la carpeta R, pero en este caso, su nombre sera \textbf{nombre\_del\_archivo\textbf{.R}d}.
En la carpeta data se encontrar\'an todos los ficheros de datos que se deseen incorporar al paquete. La extensi\'on 
de estos ficheros ser\'an \textbf{\textbf{.R}Data} o \textbf{\textbf{.R}da}.
Por \'ultimo, en la carpeta \textbf{inst} estar\'an todos aquellos ficheros que deseemos que se instalen con el paquete.


\subsection{Requisitos previos}

\begin{itemize}
    \item R Studio
    \item Paquete \textbf{devtools}
    \item Paquete \textbf{\textbf{roxygen2}} 
    \item Rtools
\end{itemize}

\subsection{Crear proyecto R}

en este caso lo podemos hacer tanto con R Studio, como se muestra a continuaci\'on,
como con \textbf{devtools::create(}\enquote*{ruta/del/paquete/nombreDelPaquete}\textbf{)}, ambas opciones nos
proporcionaran un \enquote*{esqueleto} del paquete con los archivos imprescindibles.
Una vez abierto R Studio, lo primero que debemos hacer es crear un nuevo proyecto para
ellos vamos a: 
\begin{center}
    \textbf{file > new project > new directory > r package} 
\end{center}

En la casilla \textbf{Package name:} especificamos el nombre de nuestro proyecto; en la casilla
\textbf{Create package based on source files} podemos a\~nadir (picando en el bot\'on \textbf{Add}) los
archivos \textbf{.R} que contengan nuestras funciones, si los hemos generado previamente (no es
estrictamente necesario a\~nadirlos ahora, se pueden a\~nadir posteriormente).
Por \'ultimo, indicamos el directorio ra\'iz en el que se situar\'a el subdirectorio en que estar\'an
contenidos los archivos de nuestro proyecto. 
Al pinchar en el bot\'on \textbf{Create Project} se crear\'a un directorio en la ubicaci\'on indicada con la
estructura necesaria para que R pueda construir el paquete.

Una vez ajustada la configuración, pinchamos en \textbf{Install and Restart} de la pestaña \textbf{Build}, esto hará que \textbf{roxygen2} genere automáticamente el archivo \textbf{.rd} de ayuda de nuestra función en la carpeta \textbf{man}.

\subsection{Documentaci\'on}

La documentaci\'on es una de las partes m\'as importantes del paquete en R, esta sirve para
ayudar a los usuarios a usar el paquete, para los desarrolladores que quieran extenderlo e
incluso para el propio creador del paquete, para en un futuro poder recordar para que serv\'ian
sus funciones.
Para que R pueda generar autom\'aticamente el archivo de ayuda para nuestras funciones
vamos a hacer uso del paquete \textbf{roxygen2}, el cual se basa en la inclusi\'on de una serie de tags relacionados con el objetivo y descripci\'on de la funci\'on y/o paquete 

Figura 3

Para ello debemos configurar las \textbf{Build Tools} de nuestro proyecto:

Figura 4

\subsubsection{Archivo DESCRIPTION}

En primer lugar, el archivo \textbf{DESCRIPTION} es lo que indica a R Studio que el directorio que lo
contiene es un paquete. Cuando creamos un proyecto e indicamos a R Studio que se va a
tratar de un paquete, este autom\'aticamente genera una plantilla del archivo \textbf{DESCRIPTION}
con los campos m\'inimamente necesarios. De igual forma, si creamos el paquete haciendo uso de la herramienta \textbf{devtools}, obtendremos
una plantilla similar.

Figura 5

Este archivo usa un formato simple denominado Debian control format (DCF) cuya estructura
es muy simple, cada l\'inea consiste en un campo y un valor separados por \enquote*{:}, cuando el valor
ocupa m\'as de una l\'inea hay que tabular las l\'ineas, como se puede ver en el campo
\textbf{DESCRIPTION} en la figura 5.
A parte de los campos de la plantilla existen otros m\'as a tener en cuenta como son \textbf{Imports}
y \textbf{Suggests}. A continuaci\'on, se proceder\'a a detallar el prop\'osito y/o contenido de cada uno
de los principales campos del archivo.
\textbf{Package}: no es m\'as que el nombre que recibir\'a nuestro paquete.
\textbf{Type}: Sirve para indicar a R Studio que se trata de un paquete (este campo no aparece al
crear nuestro paquete con \textbf{devtools})
\textbf{Title}: Comentario en una l\'inea que resume la funcionalidad/cometido de nuestro paquete, es
usado por R Studio en la lista de paquetes disponibles, por lo que es recomendable que no sea demasiado largo (65 caracteres m\'aximo).

\textbf{Version}: Indica la versi\'on de nuestro paquete, est\'a formada por al menos dos enteros
separados por “.” o “-” , como por ejemplo \textbf{1.0} o \textbf{0.7.2-5}. 

\textbf{Autor}: En este caso bastar\'ia con indicar el nombre del autor o autores del paquete.

\textbf{Maintainer}: Mantenedor del paquete.
Un paquete debe tener al menos un autor y un mantenedor, aunque pueden ser la misma
persona.

\textbf{Description}: Como su propio nombre indica, es una descripci\'on m\'as detallada del cometido
de nuestro paquete, en este caso puede ocupar varias l\'ineas de hasta 80 caracteres cada
una.
\textbf{License}: este campo puede ser una abreviaci\'on de una licencia \textbf{open source} est\'andar como
puede ser GPL-2 o BSD, o un puntero a un fichero \textbf{LICENSE} con m\'as informaci\'on.

\textbf{Imports y Suggests}: ambos sirven para indicar las dependencias de nuestro paquete con
respecto a otros paquetes.

En caso de necesitar una versi\'on espec\'ifica de un paquete lo podemos especificar entre
par\'entesis:

\textbf{Imports} indica que un paquete es necesario para que nuestro paquete funcione, de hecho,
en el momento que el usuario instale nuestro paquete, si los paquetes declarados en \textbf{Imports}
no est\'an ya instalados, se instalaran autom\'aticamente.

\textbf{Suggests} indica que nuestro paquete puede hacer uso de otro paquete, pero no es
estrictamente necesario, en este caso los paquetes sugeridos no son instalados de forma
autom\'atica y es el desarrollador el responsable de realizar la comprobaci\'on.

La forma m\'as f\'acil de a\~nadir un paquete a \textbf{Imports} y \textbf{Suggests} es mediante el uso de
\textbf{devtools}, otra forma es haci\'endolo a mano en el archivo \textbf{DESCRIPTION}.

Figura 11

\subsubsection{Documentaci\'on de objetos}

En este caso vamos a detallar la documentaci\'on de los distintos elementos que componen el
paquete.
Este tipo de documentaci\'on es accesible a trav\'es de \textbf{?} o \textbf{help()} y funciona como un
diccionario, permitiendo al usuario obtener informaci\'on sobre una funci\'on, paquete o conjunto
de datos.

\textbf{Crear la documentaci\'on}
Para crear esta documentaci\'on, lo primero que debemos hacer es a\~nadir comentarios
\textbf{roxygen} a nuestras archivos \textbf{.R}, como, por ejemplo

Figura 12

Una vez completado esto, presionamos \textbf{Ctrl/Cmd + Shift + B}, esto reconstruir\'a
completamente el paquete, actualizar\'a la documentaci\'on, reiniciar\'a R y recargar\'a nuestro
paquete. \\

\textbf{Comentarios roxygen} \\
Estos comentarios comienzan con \textbf{\#'} y se colocan delante de cada funci\'on formando lo que
se llama un \textbf{bloque}. Estos bloques se dividen en \textbf{tags} (@NombreDelTag), el contenido de un
\textbf{tag} se extiende desde el final del nombre del tag hasta el inicio del siguiente o el final del
bloque.
\textbf{Nota}: dado que el car\'acter \textbf{@} tiene un significado especial para \textbf{roxygen}, en caso de querer
introducir un \textbf{@} en el texto (para un email, por ejemplo), deber\'iamos escribir \textbf{@@}. Lo mismo
ocurre con “%” y “\”, los cuales tienen un significado especial en Latex, para usarlos debemos
escribir “\%” y “\\” respectivamente.
Adem\'as, cada bloque incluye texto previo al primer tag, esto se denomina la introducci\'on y
se estructura de la siguiente forma:

\begin{itemize}
    \item La primera frase se corresponde con el t\'itulo de la documentaci\'on, es lo primero que
se ve cuando consultamos la ayuda de un paquete o funci\'on. Debe constar de solo
una l\'inea y acabar en punto.
    \item El segundo p\'arrafo es la descripci\'on, se muestra inmediatamente despu\'es del t\'itulo y
debe dar una breve descripci\'on de lo que hace la funci\'on.
    \item Por \'ultimo, el resto de p\'arrafos, si los hubiera, corresponden a una descripci\'on m\'as
detallada del funcionamiento de la funci\'on. Tambi\'en podemos usar el tag \textbf{@section}
para dividir los detalles de la funci\'on en distintas secciones.
\end{itemize}

El t\'itulo y la descripci\'on son obligatorios, mientras que los detalles son opcionales.
\textbf{text}{Nota}: cada l\'inea de \textbf{roxygen} no debe superar los 80 caracteres y es recomendable tabular
las l\'ineas para facilitar la lectura.

Existen m\'as tags que son situacionales, es decir, est\'an basados en el objeto el cual estamos
documentando. A continuaci\'on, vamos a estudiar estos tags en funci\'on del objeto al que
documentan. \\
\textbf{Documentando funciones} \\
Adem\'as del bloque introductorio y los tags ya mencionados, las funciones cuentan con una
serie de tags propios:
\begin{itemize}
    \item \textbf{@param} compuesto por un nombre y una descripci\'on. Sirve para describir los
par\'ametros de entrada de la funci\'on, su nombre, su tipo y para qu\'e se utilizan.
La descripci\'on debe comenzar con letra may\'uscula y acabar en un punto, puede
abarcar varias l\'ineas y no debe quedar ning\'un par\'ametro sin documentar.
Se pueden documentar al mismo tiempo par\'ametros similares y/o del mismo tipo
separando sus nombres con comas.
    \item \textbf{@example}s sirve para proveer c\'odigo en R que muestre un ejemplo de c\'omo funciona
la funci\'on. El c\'odigo debe funcionar sin errores. Esto es importante ya que muchos
usuarios miran primero los ejemplos.
    \item \textbf{@return} descripci\'on de la salida de la funci\'on, no siempre es necesario, pero es
bueno incluirlo si, por ejemplo, la funci\'on devuelve distintos tipos de salida en funci\'on
de los par\'ametros de entrada o si estamos devolviendo un objeto S3, S4 o RC.
\end{itemize}

\textbf{Documentando paquetes}\\

Podemos usar \textbf{roxygen} para crear una p\'agina de ayuda propia del paquete, no asociada a
una funci\'on en particular. Esta p\'agina es accesible mediante package?nombreDelPaquete y
puede ser usada para describir los componentes m\'as importantes del paquete o las
dependencias que tiene, por ejemplo.
En este caso, dado que no se corresponde con un objeto en concreto, debemos etiquetarlo
manualmente como \textbf{@docType package} y \textbf{@name} nombreDelPaquete y poner un \textbf{NULL} al
final.

Figura 13\\

\subsubsection{Datos externos}

Hay tres maneras principales de incluir datos en un paquete, dependiendo de qu\'e queramos
hacer con ellos o qui\'en debe usarlos:

\begin{itemize}
    \item Si se quiere almacenar datos binarios y que est\'en disponibles para los usuarios, se
deben poner en el directorio data/. Este es el mejor sitio para los datasets.
    \item Si lo que se quiere es almacenar datos de an\'alisis, pero que los dem\'as usuarios no
los tengan disponibles, se deben colocar en R/sysdata\textbf{.R}da. En este caso es el mejor
sitio para los datos que necesiten nuestras funciones.
    \item Por \'ultimo, si queremos almacenar datos raw, se deben almacenar en el directorio
inst/data.
\end{itemize}

Si \textbf{DESCRIPTION} incluye LazyData: true, entonces los datasets se cargar\'an de forma
“Lazy”. Es decir, que no ocupar\'an memoria hasta que se usen, \textbf{devtools::cr\'eate()} establece
LazyData a true autom\'aticamente.

\subsubsection{NAMESPACE}

El archivo NAMESPACE es muy importante en nuestro paquete, ya que nos va a ayudar a
encapsular nuestro paquete y que no falle por las dependencias que pueda tener de otros, ni
interfiera en otros que usen el nuestro si lo actualizamos.
Como su nombre indica, NAMESPACE proporciona “espacios” para “nombres”. Por ejemplo,
cuando importamos dos paquetes que ambos contienen una funci\'on con el mismo nombre,
podemos desambiguarla con el operador “::”. Por ejemplo, si usamos los paquetes plyr ys
Hmisc, ambos tienen la funci\'on sumarize(), por lo que, dependiendo de cu\'al queramos usar,
usaremos \textbf{plyr::sumarize()} o \textbf{Hmisc::sumarize()}.
NAMESPACE hace nuestro paquete autocontenido, tanto con los \textbf{Imports}, como con los
exports. Los \textbf{Imports}, definen c\'omo una funci\'on de un paquete encuentra una funci\'on de
otro. Los \textbf{exports}, nos ayudan a evitar los conflictos con otros paquetes especificando qu\'e
funciones se pueden usar fuera de nuestro paquete.

\textbf{El archivo NAMESPACE}
Cada l\'inea contiene una directriz: S3method(), export(), exportClasses(), adem\'as de otros.
Cada una de estas directivas describe un objeto de R, que nos indica si se exporta desde
nuestro paquete para ser usado por los dem\'as o si se importa para usarlo de forma local.
En total hay ocho directrices. Cuatro de ellas describen las exportaciones:

\begin{itemize}
    \item export(): exporta una funci\'on.
    \item exportPattern(): exporta todas las funciones que coinciden con un patr\'on.
    \item exportClasses(), exportMethods(): exporta las clases S4 y sus m\'etodos.
    \item S3method(): exporta los m\'etodos de S3.
\end{itemize}

Y las cuatro restantes describen las importaciones:
\begin{itemize}
    \item import(): importa todas las funciones de un paquete.
    \item importFrom(): importa las funciones seleccionadas.
    \item importClassesFrom(), importMethodsFrom(): importa las clases S4 y sus
m\'etodos.
    \item useDynLib(): importa una funci\'on de C.
\end{itemize}

Generar el namespace con \textbf{roxygen2} es como generar la documentaci\'on para las funciones,
usamos los bloques y tags, a\~nadiendo los comentarios al \textbf{.R} correspondiente, ejecutamos
\textbf{devtools}::document() (o, Ctrl + Shift +D)y comprobamos que el archivo NAMESPACE se
haya modificado correctamente.
Ejemplo de archivo NAMESPACE generado por \textbf{roxygen2}:

Figura 17

\textbf{Exports}
Para que una funci\'on sea usable fuera de nuestro paquete debemos exportarla, si creamos
nuestro paquete con \textbf{text}{devtools::create()}, se cre\'o un NAMESPACE temporal que exporta todo
lo que haya en nuestro paquete que no comience con un punto “.”. Si planeamos distribuir
nuestro paquete, debemos ser cuidadosos con lo que exportamos para reducir en lo posible
los conflictos con otros paquetes.
Para exportar un objeto, se debe poner “@export” en el bloque roxygen. Por ejemplo:

Figura 18

Las funciones que se exporten deben estar documentadas.

\textbf{Imports}
NAMESPACE tambi\'en controla las funciones externas que pueden ser usadas en nuestro
paquete sin usar “::”. Esto puede confundirnos si recordamos que el archivo \textbf{DESCRIPTION}
(a trav\'es del campo Import) tambi\'en est\'a involucrado en los paquetes que importamos al
nuestro. Sin embargo, existe una diferencia muy clara, el campo Import del archivo
\textbf{DESCRIPTION} solo comprueba que el paquete est\'a instalado cuando se instala nuestro
paquete, pero no hace disponibles las funciones de dicho paquete, para ellos debemos usar
las directivas import de NAMESPACE.
Por otro lado, \textbf{text}{Depends}, se encarga de que si nuestro paquete esta adjuntado (attached),
todos los paquetes que est\'en listados en \textbf{text}{Depends}, tambi\'en se adjunten y, en el caso de que
nuestro paquete este cargado (loaded), estos paquetes lo est\'en tambi\'en.
Lo m\'as recomendable es que el paquete este importado en \textbf{DESCRIPTION} y no en el
NAMESPACE y referenciar a las funciones haciendo uso de “::”, esto puede suponernos m\'as
trabajo pero har\'a nuestro c\'odigo m\'as f\'acil de leer, entender y mantener. El contrario no se
cumple, todo paquete importado en NAMESPACE, debe estar tambi\'en en \textbf{DESCRIPTION}.

\subsection{Comprobando el paquete}

Una parte importante del desarrollo de un paquete es comprobar que el c\'odigo y la
documentaci\'on no presenten problemas, especialmente si se planea distribuir p\'ublicamente
el paquete. Para ello contamos con la herramienta R CMD check.
R CMD check es el nombre del comando que se usa desde una terminal, pero en nuestro
caso vamos a usar la herramienta que nos proporciona devtools, en concreto,
devtools::check() o, equivalentemente, Ctrl + Shift + E en R Studio.
Esta herramienta se encarga de que toda la documentaci\'on del paquete este actualizada, ya
que ejecuta devtools::document() de manera autom\'atica.
Encapsula el paquete antes de realizar las comprobaciones. Esto asegura que la
comprobaci\'on del paquete se hace desde un estado limpio ya que el encapsulamiento no
contiene ning\'un posible archivo temporal que se haya podido acumular y que pueda interferir
en las comprobaciones.
La mejor forma, aunque tediosa, de comprobar el paquete es la siguiente:

\begin{itemize}
    \item Ejecutar devtools::check() o Ctrl + Shift + E en R Studio.
    \item Arreglar el primer problema.
    \item Repetir los dos anteriores hasta que no haya m\'as problemas.
\end{itemize}

R CMD check devuelve tres tipos de problemas distintos:

\begin{itemize}
    \item Error: Problema grave que deber ser arreglado, aunque no planeemos subir el
paquete a CRAN.
    \item Warning: Problema medio que debe ser arreglado en caso de que planeemos subir
el paquete a CRAN.
    \item Note: Problema medio que deber\'iamos arreglar en caso de querer subir el paquete a
CRAN incluso si se trata de un falso positivo. En el caso de que no tengamos ninguna
nota, el proceso de subida a CRAN ser\'a autom\'atico sin intervenci\'on humana. Si no
es posible eliminar una nota, debemos describir por qu\'e no supone un problema.
\end{itemize}


\subsection{Liberando el paquete}

Si queremos que nuestro paquete tenga cierta trascendencia en la comunidad de R,
necesitamos subirlo a CRAN, ya que la mayor\'ia de usuarios descargan e instalan sus
paquetes desde CRAN.
A continuaci\'on, se detallar\'an las buenas pr\'acticas a la hora de distribuir nuestro paquete a
en CRAN.


\textbf{Comentarios CRAN.}
Debemos recordar que CRAN est\'a compuesto por voluntarios que dedican parte de su tiempo
libre a revisar los paquetes que se suben, por tanto, cuanto m\'as f\'acil les hagamos el trabajo,
m\'as posibilidades hay de que nuestro paquete sea aceptado sin ning\'un problema.
Por ello, es recomendable a\~nadir a nuestro paquete una serie de comentarios que faciliten la
tarea de revisi\'on, estos comentarios ir\'an en un archivo llamado cran-comments.md, veamos
un ejemplo (ggplot2):

Figura 24

\begin{itemize}

    \item Test environments: indica en qu\'e plataformas se ha comprobado el paquete. Dado
que R est\'a en continua evoluci\'on, es recomendable usar siempre la \'ultima versi\'on de
desarrollo del lenguaje y comprobar que nuestro paquete funciona correctamente en
al menos dos plataformas (Windows, Linux o MAC).
    \item Check results: contendr\'a una lista de errores, advertencias o notas, aunque debemos
asegurarnos de que no hay errores o advertencias (ya que, de ser as\'i, no ser\'a
aceptado en CRAN).
En el caso de que hubiera alguna nota que no hayamos podido arreglar, incluir\'iamos
el mensaje obtenido con la nota y una explicaci\'on de por qu\'e no est\'a arreglada o
porque consideramos que no afecta al correcto funcionamiento del paquete.
Es importante eliminar las notas, ya que de esta forma es m\'as probable que nuestro
paquete no requiera revisi\'on humana y el proceso de subida sea m\'as r\'apido.
    \item Reverse dependencies: sirve para indicar que si hay errores de dependencias con
otros paquetes en el caso de que estemos subiendo una nueva versi\'on de nuestro
paquete y los cambios hayan provocado errores en las dependencias con otros
paquetes. En ese caso, podemos contactar con el autor del paquete para indicar lo
que ocurre y buscar una soluci\'on.
\end{itemize}

Las pol\'iticas en CRAN.
Existen ciertas pol\'iticas que debemos cumplir a la hora de subir nuestro paquete a CRAN,
estas son comprobadas manualmente. Los voluntarios de CRAN comprueban que nuestro
paquete cumple estas pol\'iticas rigurosamente, especialmente si es la primera vez que lo
subimos.
Algunos problemas comunes son:
\begin{itemize}
    \item Ausencia del email del mantenedor del paquete. 
    \item Es importante dejar claro los propietarios del copyright y si hemos usado c\'odigo
externo, comprobar que la licencia es compatible.
    \item Paquetes que no funcionan en al menos dos plataformas no ser\'an considerados.
    \item No hacer cambios externos sin permiso del usuario: cambiar opciones, instalar
paquetes, abrir software externo, cerrar R, etc.
    \item No subir actualizaciones del paquete muy frecuentemente. La pol\'itica sugiere hacerlo
cada 1 o 2 meses como mucho.
\end{itemize}

Archivos importantes.
Una vez que nuestro paquete ya est\'a listo para subirlo, debemos tener en cuenta dos archivos
importantes m\'as, README.md, el cual proporciona una descripci\'on de nuestro paquete y su
funcionamiento y uso, y NEWS.md, el cual describe los cambios con respecto a la versi\'on
anterior de nuestro paquete. Dado que estos dos archivos no son soportados por CRAN no
hay que a\~nadirlos a la hora de construir el paquete (devtools::use\_build\_ignore("NEWS.md"),
devtools::use\_build\_ignore("README.md"))

Lanzamiento.
Lleg\'o la hora de construir nuestro paquete y subirlo a CRAN. Para construir nuestro paquete
usamos devtools::release(). Esto se encarga de volver a comprobar el paquete una \'ultima
vez y nos hace una serie de preguntas acerca de buenas pr\'acticas a modo de comprobaci\'on,
devtools::release(), autom\'aticamente, sube el paquete a CRAN.
Una vez subido, recibiremos un email de confirmaci\'on a los pocos minutos pidi\'endonos
confirmar nuestro email. Despu\'es los voluntarios de CRAN comprobar\'an el paquete y nos
devolver\'an los resultados, esto normalmente tarda desde 24h hasta 5 d\'ias.
En caso de fallo, este vendr\'a descrito en los resultados de la comprobaci\'on, en este caso,
debemos arreglar el fallo, incluir en cran-coments un comentario indicando que vamos a
hacer una resubida, actualizar las dependencias en caso de que sea necesario y ejecutar
devtools::submit\_cran() para evitar contestar de nuevo todas las preguntas de
devtools::release().
Una vez que CRAN ha aceptado nuestro paquete, construir\'a el paquete para las distintas
plataformas, esto puede dar lugar a nuevos errores que debemos arreglar en un parche y
volver a subir.