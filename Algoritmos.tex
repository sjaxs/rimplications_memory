\section{Algoritmos}
\subsection{IO library}

\lstinputlisting{I_O.R}

%Estudio de FCA
\subsection{FCA}





%Estructura de datos
\subsection{Estructura de datos}

Cuando se comienza a realizar un algoritmo que se necesita que 
sea r\'apido y est\'e optimizado, una de las cosas m\'as importantes 
es decidir qu\'e estructura de datos se va a utilizar.
Ya que, dependiendo de qu\'e queramos hacer, elegir bien, nos puede 
ayudar a ganar tiempo en el algoritmo.
\\
\\
Por ello, para la gran mayor\'ia de algoritmos que vamos a realizar 
en este TFG se va a utilizar la estructura ItemMatrix, perteneciente al paquete 
arules.

 %Librería básica
\subsection{Librer\'ia b\'asica para sistemas de implicaciones}


%Conjuntos aleatorios
\subsection{Random Implications Generator}


%Data mining
\subsection{Data mining}




\newpage
\thispagestyle{empty}
\mbox{}

\newpage