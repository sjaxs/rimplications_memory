\section{Algoritmos}
\subsection{IO library}

\lstinputlisting{I_O.R}

%Estudio de FCA
\subsection{FCA}




\clearpage
%Estructura de datos
\subsection{Estructura de datos}

Cuando se comienza a realizar un algoritmo que se necesita que 
sea r\'apido y est\'e optimizado, una de las cosas m\'as importantes 
es decidir qu\'e estructura de datos se va a utilizar.
Ya que, dependiendo de qu\'e queramos hacer, elegir bien nos puede 
ayudar a que el algoritmo se ejecute en el menor tiempo posible.

Por ello, en la gran mayor\'ia de algoritmos que se van a realizar 
en este TFG se va a utilizar la estructura rules, formada por varios 
elementos con la estructura ItemMatrix, perteneciente al paquete 
arules.
\\

La estructura de datos rules est\'a formada por cuatro slots, es decir, por 
cuatro campos de informaci\'on diferentes:

\begin{itemize}
    \item \textbf{lhs:}
    Lista de objetos de la clase ItemMatrix, en la cual se profundizar\'a m\'as 
    adelante. En este slot encontraremos todas las partes izquierdas de las 
    implicaciones, cada una de ellas con varios slots m\'as debido a que son de la 
    clase ItemMatrix. 

    \item \textbf{rhs:}
    De nuevo, es una lista de objetos de la clase ItemMatrix. S\'olo que en este 
    slot encontraremos las partes derechas de las implicaciones.

    \item \textbf{quality:}
    Objeto de la clase data.frame. En \'el se puede consultar la confianza de cada 
    una de las reglas. La confianza se define como el porcentaje de transacciones 
    que dan soporte a la regla con respecto a todas las transacciones que dan 
    soporte al cuerpo de regla. Una transacci\'on da soporte al cuerpo de regla si 
    contiene todos los items del cuerpo de regla.

    \item \textbf{info:}
    Lista con varios campos con informaci\'on del conjunto de implicaciones. Se puede 
    encontrar el nombre del dataset al que pertenecen, el n\'umero de transacciones, 
    el soporte o la confianza.

\end{itemize}


Como se ha visto en los apartados lhs y rhs, ambos est\'an formados por una lista de 
ItemMatrix, por lo que a continuaci\'on se va a detallar dicha clase, con todos sus slots.

La clase ItemMatrix es la estructura b\'asica para las transacciones, los 
conjuntos de elementos y las reglas.
La clase contiene una representaci\'on de una matriz dispersa de elementos 
y las etiquetas correspondientes a dichos elementos.

Los conjuntos de elementos se representan como matrices binarias dispersas. 
Si se trabaja con varios ItemMatrix al mismo tiempo, la codificaci\'on 
en los diferentes ItemMatrix es importante, ya que ambos elementos deben tener 
las mismas etiquetas para poder trabajar con ellos.

Cada ItemMatrix tiene la siguiente informaci\'on:
\begin{itemize}

    \item \textbf{data:}
    Objeto de la clase ngCMatrix que almacena las ocurrencias de elementos en 
    representaci\'on dispersa. Hay que tener en cuenta que ngCMatrix est\'a orientado 
    a columnas, e ItemMatrix a filas, y cada una de dichas filas representa un elemento.
    Como resultado, el ngCMatrix en este caso siempre ser\'a una versi\'on traspuesta 
    de la matriz de incidencia binaria en ItemMatrix.


    \item \textbf{itemInfo:}
    Es un data.frame que contiene vectores de longitud igual al n\'umero de elementos que 
    hay en el conjunto. Si no est\'a vac\'io, el primer elemento del data.frame debe tener 
    \textit{labels} como nombre, y contener un vector de car\'acteres con las etiquetas de los 
    elementos utilizados para representar dicho elemento. Adem\'as de estas etiquetas, el 
    data.frame puede contener vectores con nombre arbitrario, para representar, por ejemplo, 
    nombres de variables y valores que se usaron para crear los elementos binarios o 
    informaci\'on de categor\'ia jer\'arquica asociada con cada etiqueta.


    \item \textbf{itemsetInfo:}
    Es un data.frame que puede contener informaci\'on adicional para las filas en la matriz.

\end{itemize}


Veamos un ejemplo:

Si tenemos las siguientes implicaciones guardadas en la variable \textbf{rules}:
\begin{verbatim}
    a -> b
    c -> a
    d -> e
\end{verbatim}


Si ejecutamos el comando \textbf{rules@lhs@data} vamos a obtener una matriz dispersa de la clase 
ngCMatrix en la que podremos observar qu\'e elementos se encuentran en la parte izquierda de 
nuestras implicaciones.

\begin{verbatim}
    5 x 3 sparse Matrix of class "ngCMatrix"
    [1,] | . .
    [2,] . . .
    [3,] . | .
    [4,] . . |
    [5,] . . .
\end{verbatim}

Es decir, las filas son los diferentes elementos que tenemos en todas las reglas, y las 
columnas son cada una de las implicaciones.


Al ejecutar \textbf{rules@lhs@itemInfo}, vamos a obtener el nombre de cada uno de los elementos que 
tiene nuestro conjunto de implicaciones. 

\begin{verbatim}
    labels
    1      a
    2      b
    3      c
    4      d
    5      e
\end{verbatim}


Por \'ultimo, si ejecutamos \textbf{rules@lhs@itemsetInfo}, se obtiene el n\'umero de columnas y filas 
que tiene el data.frame de info. 
Como en este caso no tiene ninguna informaci\'on, obtenemos lo siguiente:
\begin{verbatim}
    data frame with 0 columns and 0 rows
\end{verbatim}


Esta estructura de datos, es bastante \'util para trabajar con conjuntos, ya que se pueden 
realizar todas las operaciones de forma eficiente, como por ejemplo, uni\'on o intersecci\'on.
Adem\'as, si se quiere obtener un conjunto de implicaciones de un dataset, al usar la funci\'on 
apriori del paquete arules, nos devolver\'a un conjunto en un ItemMatrix, por lo que nos va a 
resultar m\'as sencillo trabajar con este tipo de estructura, ya que es la m\'as 
extendida a nivel general. Por lo que a la hora de publicar el paquete de funciones, cuando otras 
personas quieran usar nuestros algoritmos, lo podr\'an realizar con facilidad, ya que se ajusta 
a c\'omo se trabaja en la actualidad en an\'alisis de datos.

As\'i que, debido a la estructura de esta clase, la facilidad de uso, y la adecuaci\'on a 
los datos que se van a utilizar a lo largo de este TFG, se ha elegido la estructura rules, compuesta de ItemMatrix 
como la principal a usar.


\clearpage








%Librer\'ia b\'asica
\subsection{Librer\'ia b\'asica para sistemas de implicaciones}
Se va a intentar que este paquete de funciones pueda perdurar en el tiempo, y para ello, 
se a realizado una librer\'ia b\'asica para intentar modularizar el c\'odigo y que sea lo 
m\'as legible posible a la hora de poder comprenderlo.

En esta librer\'ia b\'asica se encuentran diferentes funciones que se usan bastante, a continuaci\'on 
se van a detallar una a una.




%%% initialize.setOfAttributes %%%

\subsubsection{Descripci\'on}

Esta funci\'on se utilizar\'a para inicializar un conjunto de datos, es decir, le vamos a pasar un subconjunto 
de todos los atributos que componen a un grupo de implicaciones, adem\'as de todos los atributos. Dentro de esta 
se va a utilizar la funci\'on encode para convertir el subconjunto de atributos en un objeto itemMatrix y de esta 
forma poder trabajar con ellos en los diferentes algoritmos que se van a desarrollar a lo largo de este TFG.


\subsubsection{Ejemplo}
Para que se entienda mejor, se va a plantear un ejemplo. 

Si tenemos el conjunto de implicaciones siguiente: 



\subsubsection{C\'odigo}














\clearpage

%Conjuntos aleatorios
\subsection{Random Implications Generator}



\clearpage

%Data mining
\subsection{Data mining}




\clearpage



\newpage
\thispagestyle{empty}
\mbox{}

\newpage