\section{Algoritmos}
\subsection{IO library}

\lstinputlisting{I_O.R}

%Estudio de FCA
\subsection{FCA}





%Estructura de datos
\subsection{Estructura de datos}

Cuando se comienza a realizar un algoritmo que se necesita que 
sea r\'apido y est\'e optimizado, una de las cosas m\'as importantes 
es decidir qu\'e estructura de datos se va a utilizar.
Ya que, dependiendo de qu\'e queramos hacer, elegir bien nos puede 
ayudar a que el algoritmo se ejecute en el menor tiempo posible.

Por ello, en la gran mayor\'ia de algoritmos que se van a realizar 
en este TFG se va a utilizar la estructura ItemMatrix, perteneciente al paquete 
arules.
\\

La clase ItemMatrix es la estructura b\'asica para las transacciones, los 
conjuntos de elementos y las reglas.
La clase contiene una representaci\'on de una matriz dispersa de elementos 
y las etiquetas correspondientes a dichos elementos.

Los conjuntos de elementos se representan como matrices binarias dispersas. 
Si se trabaja con varios ItemMatrix al mismo tiempo, la codificaci\'on 
en los diferentes ItemMatrix es importante, ya que ambos elementos deben tener 
las mismas etiquetas para poder trabajar con ellos.

Cada ItemMatrix tiene la siguiente informaci\'on:
\begin{itemize}

\item data:
Objeto de la clase ngCMatrix que almacena las ocurrencias de elementos en 
representaci\'on dispersa. Hay que tener en cuenta que ngCMatrix est\'a orientado 
a columnas, e ItemMatrix a filas, y cada una de dichas filas representa un elemento.
Como resultado, el ngCMatrix en este caso siempre ser\'a una versi\'on transpuesta 
de la matriz de incidencia binaria en ItemMatrix.


\item itemInfo:
Es un data.frame que contiene vectores de longitud igual al n\'umero de elementos que 
hay en el conjunto. Si no est\'a vac\'io, el primer elemento del data.frama debe tener 
\textit{labels} como nombre, y contener un vector de car\'acteres con las etiquetas de los 
elementos utilizados para representar dicho elemento. Adem\'as de estas etiquetas, el 
data.frame puede contener vectores con nombre arbitrario, para representar, por ejemplo, 
nombres de varialbes y valores que se usaron para crear los elementos binarios o 
informaci\'on de categor\'ia jer\'arquica asociada con cada etiqueta.


\item itemsetInfo:
Es un data.frame que puede contener informaci\'on adicional para las filas en la matriz.

\end{itemize}



Esta estructura de datos, es bastante \'util para trabajar con conjuntos, ya que se pueden 
realizar todas las operaciones de forma eficiente, como por ejemplo, uni\'on o intersecci\'on.
Adem\'as, si se quiere obtener un conjunto de implicaciones de un dataset, al usar la funci\'on 
apriori del paquete arules, nos devolver\'a un conjunto en un ItemMatrix, por lo que nos va a 
resultar m\'as sencillo trabajar con este tipo de estructura en general, ya que es la m\'as 
extendida a nivel general. Por lo que a la hora de publicar el paquete de funciones, cuando otras 
personas quieran usar nuestros algoritmos, lo podr\'an realizar con facilidad, ya que se ajusta 
a c\'omo se trabaja ahora en an\'alisis de datos.

As\'i que, debido a la estructura de esta clase, la facilidad de uso, y la adecuaci\'on a 
los datos que se van a utilizar a lo largo de este TFG, se ha elegido la estructura ItemMatrix 
como la principal a usar. 















%Librería básica
\subsection{Librer\'ia b\'asica para sistemas de implicaciones}


%Conjuntos aleatorios
\subsection{Random Implications Generator}


%Data mining
\subsection{Data mining}




\newpage
\thispagestyle{empty}
\mbox{}

\newpage