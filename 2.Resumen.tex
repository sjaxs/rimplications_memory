\setcounter{page}{5}
{

\textbf{Resumen:}

Este Trabajo Fin de Grado (TFG) tiene como objetivo principal, la creaci\'on y publicaci\'on de una librer\'ia 
en R para el An\'alisis Formal de Conceptos. Se ha realizado en la modalidad de trabajo en equipo de dos personas. 
Este trabajo se ha dividido en dos partes, teniendo partes en com\'un y partes diferenciadas en ambos. La parte conjunta 
consta de un estudio avanzado de R, as\'i como un estudio de c\'omo realizar y distribuir un paquete en R para el repositorio 
CRAN, manual de usuario y documentaci\'on del paquete completo.

La primera parte de este trabajo, se basa fundamentalmente en la creaci\'on de funciones para su posterior utilizaci\'on. Por lo que, 
en este apartado se ha desarrollado una librer\'ia b\'asica, un generador aleatorio de sistemas de implicaciones y uno de contextos, y 
por \'ultimo, una librer\'ia de funciones para el An\'alisis Formal de Conceptos, en la que se encontrar\'an las funciones m\'as importantes 
de este campo.

En la segunda parte, podremos encontrar la implementaci\'on de las reglas de la L\'ogica de Simplificaciones, algoritmos desarrollados para 
el c\'alculo de cierres y eliminaci\'on de redundancias, algoritmos dedicados al c\'alculo de claves y generadores minimales y
algoritmos dedicados al c\'alculo de bases de sistemas implicacionales.

Este proyecto es una versi\'on inicial para el paquete anteriormente citado, ya que con el paso del tiempo se ir\'a mejorando y ampliando como 
el departamento de Matem\'atica Aplicada de la UMA considere oportuno.



\bigskip

\textbf{Palabras clave:}

L\'ogica de Simplificaciones, Descubrimiento de conocimiento, 
An\'alisis Formal de Conceptos, Contexto formal.

\clearpage

\textbf{Abstract:}

This Degree Thesis (TFG) aims to create and publish a R package for Formal Concept Analysis. The TFG has been done by two people in the 
form of teamwork. The work was divided in two parts, one joint and other individually. The joint part has an advanced R research, 
a research about building and publishing a R package in CRAN, user manual and package documentation. 

The first part of this project 
is basically based in the creations of functions that will be used later. So, in this part, a basic library, a random implications system 
generator, a random context generator and FCA library, where the most important functions in this area are, have been developed. 

In the 
second part, we could find the Simplification Logic implementation, closure compute and redundancy removing algorithms, keys and minimal 
generators finding algorithms and basis compute algorithms. 

This project is an initial version of the R package that we mentioned before, 
it will be improved and extended in followings projects by Applied Mathematics department of UMA.

\bigskip

\textbf{Key words:} 

Logic of simplifications, Discovery of knowledge, 
Formal Concept Analysis, Formal Context.

}
\newpage
\thispagestyle{empty}
\mbox{}
\newpage