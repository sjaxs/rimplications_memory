\subsection{Generadores aleatorios}

A la hora de realizar pruebas con un algoritmo, lo m\'as aconsejable es que se realice con conjuntos generados de forma aleatoria, para 
que, por ejemplo, los an\'alisis de tiempo no est\'en influidos por dichos conjuntos. Actualmente, en la comunidad de FCA (Formal Concept 
Analysis) no existe nada parecido, por lo que se ha decidido realizar dos funciones que puedan ser usadas para este fin y ponerlas a 
disposici\'on de toda la comunidad junto a este paquete, para que cualquier persona que quiera, pueda utilizarlas.

As\'i que, como se ha comentado anteriormente, se han realizado dos funciones diferentes, una que genera sistemas de implicaciones 
aleatorios y otro de contextos aleatorios.

A continuaci\'on se van a detallar ambas funciones:

\subsubsection{Sistemas de implicaciones aleatorios}

    \textbf{Descripci\'on}
    
    Esta funci\'on va a generar uno o m\'as sistemas de implicaciones siguiendo una serie de par\'ametros que se pueden elegir a la 
    hora de ejecutar dicha funci\'on para obtener el conjunto como sea m\'as adecuado a lo que se desee hacer.

    Los atributos de las implicaciones van a ser del tipo a0, a1... Comenzando en 0, de forma consecutiva hasta el n\'umero que sea 
    necesario para tener todos los atributos que el usuario quiera.

    Los par\'ametros de esta funci\'on son:

    \begin{itemize}
        \item \textbf{numAttributes:}

        Aqu\'i se debe especificar cu\'antos atributos diferentes se quiere que tengan las implicaciones que se van a generar. Es un 
        par\'ametro obligatorio para la utilizaci\'on de esta funci\'on. Como se ha comentado anteriormente, los atributos serán de la 
        forma a0, a1, a2... El valor de este par\'ametro debe ser mayor que 0.

        
        \item \textbf{numDependencies:}

        Otro par\'ametro obligatorio es el número de dependencias que se quieren obtener, si se elige obtener m\'as de un archivo 
        con diferentes sistemas de implicaciones, como se ver\'a m\'as adelante, cada uno de dichos ficheros incluir\'a el n\'umero 
        de implicaciones que se le indique aqu\'i. Por supuesto, el valor de este par\'ametro tambi\'en debe ser mayor que 0.


        \item \textbf{difference:}

        Este pa\'ametro opcional indica el n\'umero m\'aximo de diferencia de elementos entre el antecedente y el consecuente. Se puede 
        insertar un n\'umero negativo, lo que indicar\'a que se desea que haya m\'as atributos en el consecuente. Por lo que si el n\'umero 
        es positivo, el antecedente tendr\'a m\'as atributos.

        En este caso, hay que tener en cuenta de que el valor absoluto del n\'umero insertado en este par\'ametro, debe ser menor 
        que el n\'umero total de atributos.


        \item \textbf{percentage:}

        \item \textbf{maxLeftSize:}

        \item \textbf{maxRightSize:}

        \item \textbf{nameExitFile:}

        \item \textbf{numFiles:}



    \end{itemize}

    \textbf{C\'odigo}


    \textbf{Ejemplo}


    

\subsubsection{Contextos aleatorios}

    \textbf{Descripci\'on}

    \textbf{C\'odigo}

    \textbf{Ejemplo}