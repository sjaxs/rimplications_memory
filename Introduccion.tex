\section{Introducci\'on}

En el presente TFG se va a realizar un paquete de funciones en R para el \'area de An\'alisis Formal de Conceptos, que se ha constituido como una herramienta formal para al an\'alisis de datos y la extracci\'on de conocimiento a partir de un conjunto de objetos y las propiedades que cumplen dichos objetos.

El director del TFG (\'Angel Mora) est\'a desarrollando junto con su grupo de la Universidad de M\'alaga (Pablo Cordero y Manuel Enciso), su investigaci\'on en el \'area descrita. Actualmente, tienen como gran objetivo el desarrollo de una librer\'ia para el lenguaje R (R package) que implemente los algoritmos m\'as importantes que han desarrollado para la manipulaci\'on del conocimiento en  An\'alisis Formal de Conceptos. Este TFG constituye la primera versi\'on del paquete que pretendemos sea referente para la manipulaci\'on de conocimiento en aplicaciones pr\'acticas. Como segundo objetivo y no menos importante, se pretende que el paquete R, tras ser publicado, act\'ue como mecanismo de difusi\'on de los trabajos publicados por el grupo. En la actualidad aumentar el n\'umero de citas de las publicaciones es prioritario. 

Por \'ultimo, ACF es probablemente la herramienta m\'as desconocida en las \'areas de ciencia de datos y de descubrimiento de conocimiento. Se pretende que, puesto que en los ambientes de data mining, machine learning, big data, etc. el lenguaje R es uno de los m\'as usados, el uso de nuestro paquete impulse el cocimiento del propio \'area de ACF.

Al tratarse de un trabajo de fin de grado en la modalidad de equipo se ha realizado un reparto de las tareas a realizar. Dicho equipo est\'a formado por Ana Esperanza Villal\'on Mart\'in y por m\'i, Jes\'us Alberto Dom\'inguez \'Alvarez. Algunas de las partes se han desarrollado conjuntamente, mientras que otras han sido desarrolladas por cada uno de los miembros del equipo de forma individual. Dada la naturaleza de este proyecto, aunque las tareas individuales estaban bien diferenciadas, guardan relaci\'on entre s\'i.

En la parte com\'un del TFG, adem\'as de un estudio avanzado del lenguaje R, se ha realizado un estudio de c\'omo se construye un paquete en R y c\'omo se distribuye un paquete en el repositorio oficial de paquetes de R, llamado CRAN, as\'i como el manual de usuario del paquete y la documentaci\'on del mismo.

La parte espec\'ifica del TFG presentada en esta memoria consisti\'o en el desarrollo de una librer\'ia para implementar la L\'ogica de Simplificaciones, desarrollada por el equipo de M\'alaga \cite{Cordero2002}, el dise\~no de una librer\'ia para razonar con la L\'ogica de Simplificaciones y su aplicaci\'on en el c\'alculo de cierres, eliminaci\'on de redundancia, etc. desarrollo de algoritmos para la obtenci\'on de generadores minimales, claves minimales, as\'i como para la computaci\'on de bases de conjuntos de implicaciones: base directa optimal, d-base. 

En las posteriores secciones se desarrollar\'an todos los aspectos relativos al TFG. En la secci\'on 2 se presenta un estudio de la creaci\'on y distribuci\'on de un paquete R. La secci\'on 3 muestra la implementaci\'on de las reglas de la L\'ogica de Simplificaciones en R. A continuaci\'on, la secci\'on 4 contiene los algoritmos desarrollados para el c\'alculo de cierres y eliminaci\'on de redundancia. Seguidamente, en la secci\'on 5 se encuentran los algoritmos dedicados al c\'alculo de claves y generadores minimales. En la secci\'on 6 se encuentran los algoritmos dedicados al c\'alculo de bases de sistemas implicacionales. En la secci\'on 7 se encuentra el manual de usuario del paquete desarrollado. Por \'ultimo, en la secci\'on 8 se presentan las conclusiones generales del TFG. Adem\'as de estas secciones, tambi\'en se incluye la bibliograf\'ia.\\

\textbf{Motivaci\'on}

Aunque en la actualidad existe una extensa colecci\'on de software y aplicaciones desarrollados por los principales investigadores en el \'area del  An\'alisis Formal de Conceptos, ninguno de ellos ha conseguido establecerse como referencia en esta comunidad por falta de la suficiente difusi\'on y ninguna de las herramientas existentes est\'a desarrollada en R. Esto hace que la mayor\'ia de ellas sean pr\'acticamente de uso personal para el investigador que la crea. 

Por ello, con este trabajo se pretende crear una herramienta que se convierta en un est\'andar en este \'ambito, potenciado esto por el auge actual del lenguaje R y mediante la difusi\'on del paquete en el repositorio de R (CRAN).\\


\textbf{Objetivos}

El objetivo principal de este trabajo es obtener una librer\'ia en el lenguaje R, que nos permita el an\'alisis r\'apido y efectivo de un conjunto de datos y la extracci\'on de conocimiento mediante el An\'alisis Formal de Conceptos donde podemos extraer dos tipos de conocimiento, el ret\'iculo de conceptos y las implicaciones a partir de un contexto formal que no es otra cosa que una relaci\'on binaria entre los objetos y los atributos. Esta librer\'ia se pretende distribuir en el repositorio de paquetes de R, de forma que tenga acceso la comunidad de desarrolladores e investigadores en este \'ambito y de las \'areas afines nombradas anteriormente. 

El objetivo de la parte espec\'ifica es la de implementar la L\'ogica de Simplificaciones al lenguaje R y el de los algoritmos basados en dicha l\'ogica que permitan una manipulaci\'on eficiente de las implicaciones para la resoluci\'on de problemas importantes en el \'area de AFC.

\newpage