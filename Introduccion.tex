\section{Introducci\'on}

En el presente TFG se va a realizar un paquete de funciones en R para el \'area de An\'alisis de Conceptos Formales que se ha constituido como una herramienta formal para al an\'alisis de datos, que permitir\'a la extracci\'on de conocimiento a partir de un conjunto de objetos y las propiedades que cumplen dichos objetos.

El director del TFG (\'Angel Mora) est\'a desarrollando junto con su grupo de la Universidad de M\'alaga (Pablo Cordero y Manuel Enciso), su investigaci\'on en el \'area descrita. Actualmente, tienen como gran objetivo el desarrollo de una librer\'ia para el lenguaje R (R package) que implemente los algoritmos m\'as importantes que han desarrollado para la manipulaci\'on del conocimiento en  An\'alisis de Conceptos Formales. Este TFG costituye la primera versi\'on del paquete que pretendemos sea referente para la manipulaci\'on de conocimiento en aplicaciones pr\'acticas.

\textbf{Motivaci\'on}

Aunque en la actualidad existe una extensa colecci\'on de software y aplicaciones desarrollados por los principales investigadores en el \'area del  An\'alisis de Conceptos Formales, ninguno de ellos ha conseguido establecerse como referencia en esta comunidad por falta de la suficiente difusi\'on y ninguno de ellos esta desarrollada en R. Esto hace que la mayor\'ia de ellas sean pr\'acticamente de uso personal para el investigador que lo crea. 

Por ello, con este trabajo se pretende crear una herramienta que se convierta en un est\'andar en este \'ambito, potenciado esto por el auge actual del lenguaje R y mediante la difusi\'on del paquete en el repositorio de R (CRAN).


\textbf{Objetivos}

El objetivo principal de este trabajo es obtener una librer\'ia en el lenguaje R, que nos permita el an\'alisis r\'apido y efectivo de un conjunto de datos y transformarlo en conocimiento mediante el An\'alisis de Conceptos Formales y de sistemas de implicaciones. Esta librer\'ia se pretende distribuir en el repositorio de paquetes de R, de forma que tenga acceso toda la comunidad de desarrolladores e investigadores en este \'ambito.

\newpage