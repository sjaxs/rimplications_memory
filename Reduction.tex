\subsubsection{Descripci\'on} 

El concepto de clave es importante en cualquier modelo de datos y aparece con multitud de nombres en las distintas \'areas: generadores minimales en FCA o \textit{key sets} en Data Mining, por ejemplo.

El problema de la b\'usqueda de claves aparece en distintas \'areas de la computaci\'on, siendo las bases de datos el \'area en la que mayor atenci\'on se le presta. Este algoritmo desarrollado por P. Cordero et al. en  \cite{Reduction} busca resolver este problema lo m\'as eficientemente posible. Para ello trata de simplificar lo m\'aximo posible el conjunto de implicaciones del cual se quieren obtener las claves, aplicando la L\'ogica de Simplificaciones \cite{Cordero2002}.

El problema de la b\'usqueda de claves consiste en enumerar todas las claves minimales que pueden ser inferidas a partir de un conjuto de reglas. La mayor\'ia de los trabajos que se pueden encontrar en la literatura no usan un sistema axiom\'atico para inferir las claves, sino que proporcionan distintas aproximaciones a casos particulares de este problema. El objetivo de este algoritmo es el de proporcionar un m\'etodo directo basado en la L\'ogica de Simplificaciones.

\IncMargin{1em}
\begin{algorithm}[H]
    \SetKwFunction{Body}{body}
    \SetKwFunction{Core}{core}
    \SetKwFunction{Compact}{Compact}
    \SetKwFunction{EnumerateKeys}{EnumerateKeys}
    \SetAlgoLined
    % \LinesNumbered
    \DontPrintSemicolon
    \SetKw{KwOr}{or}
    \KwIn{  
        $\Omega$, a set of attributes; $\Gamma$, a set of formulas.
    }
    \KwOut{The set $K$ of all minimal keys}
    \Begin{
        \ $\Omega' := \Body{$\Omega,\Gamma$}$\;
        \ $\Gamma' := \Compact{$ \{A \cap \Omega' \to B \cap \Omega' \ | \ A \to B in \Gamma\}$}$\;
        \ $K' := \EnumerateKeys{$\Omega',\Gamma'$}$\;
        \ $K := \{ k \cup \Core{$\Omega,\Gamma$} \ | \ k \in K' \}$\;
    }%end beginre
    \caption{Reduction Method algorithm}\label{alg:3}
\end{algorithm}\DecMargin{1em}
\newpage
Se van a definir algunos conceptos que servir\'an de ayuda a la hora de entender el algoritmo desarrollado.\\

\textbf{Core y Body}

El c\'alculo de core y body que aparece a continuaci\'on se public\'o en [xx] y es una estrateg\'ia de poda (prunning) para reducir el esquema relacional en el que calcular las claves. 

La estrateg\'ia mejora todas las heur\'isitcas que aparecen en la literatura. 

Siendo \(\Omega\) un conjunto de atributos, \(\Gamma\) un conjunto de implicaciones sobre \(\Omega\), definimos el core y el body de \(\Gamma\) como:
\begin{center}
    \(core(\Omega,\Gamma) = \Omega \setminus \big(\bigcup_{\substack{A \to B \in \Gamma}} B\big) \ \ \ \ \)
    \(body(\Omega,\Gamma) = \big(\bigcup_{\substack{A \to B \in \Gamma}} A\big) \setminus (core(\Omega,\Gamma))^+\)
\end{center}
Ambos permitir\'an podar \(\Gamma\) y as\'i agilizar el proceso de obtenci\'on de las claves. Adem\'as se tiene que, si \(K\) es una clave minimal, cumple que: 
\begin{center}
    \(core(\Omega,\Gamma) \subseteq K \subseteq core(\Omega,\Gamma) \cup body(\Omega,\Gamma)\)
\end{center}

En cualquier algoritmo de c\'alculo de claves o de generadores minimales podemos podar previamente el esquema con lo que de forma evidente se mejora muy significativamente el algoritmo posterior.

\newpage
\subsubsection{C\'odigo} 
\lstinputlisting{r_code/reduction.method.R}
\newpage
\lstinputlisting{r_code/enumerate.keys.R}
\lstinputlisting{r_code/comp_unionk.R}
\newpage
\subsubsection{Ejemplo} 
Aq\'i se puede ver un ejemplo sencillo de c\'alculo de las claves minimales de un sistema implicacional:
\begin{figure}[H]
    \centering
    \includegraphics[scale=0.75]{reduction_1}
    \caption{Ejemplo Reduction}
    \label{fig:reduction_1}
\end{figure}
[VER CORE Y BODY EXPLICAR SALIDA]
\subsubsection{Comparativa/Versiones} 
La implementaci\'on inicial del algoritmo ha sido mejorada con una poda que sirve para no comprobar dos veces la misma implicaci\'on y as\'i reducir el tiempo de c\'alculo de las claves.

Como se puede observar, si el conjunto de implicaciones es peque\~no, los resultados son muy similares, pero en el caso de trabjar con 50 implicaciones, por ejemplo, la versi\'on con poda tarda alrededor de 300s menos, lo cual supone una reducci\'on de tiempo considerable.

\begin{figure}[H]
    \centering
    \includegraphics[scale=0.75]{reduction}
    \caption{Comparativa Reduction}
    \label{fig:reduction}
\end{figure}
[CAMBIAR EJEMPLO]
Obviamente, cuanto mayor sea el n\'umero de implicaciones, mayor ser\'a la ventaja del uso de la poda.