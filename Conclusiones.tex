\section{Conclusiones y trabajos futuros}

A lo largo de este TFG se ha desarrollado una librer\'ia que se utilizar\'a para el An\'alisis Formal de Conceptos, aunque la intenci\'on 
del director y de su equipo de investigaci\'on es que el paquete desarrollado sirva de referente en las \'areas en las que la manipulaci\'on 
eficiente de reglas tenga gran inter\'es, citamos bases de datos, data mining, rough set theorie, big data, etc. Este es un 
campo muy amplio, por lo que se ha requerido de un estudio amplio tanto del lenguaje R, como del campo de AFC y la base matem\'atica en la que 
est\'a fundamentado. A partir de este estudio, se procedi\'o a la implementaci\'on de los algoritmos que se han expuesto en ambas partes de 
este trabajo, as\'i como la creaci\'on de alguno de ellos, o la adaptaci\'on de los que el director ya hab\'ia comenzado.

Se quiere destacar el alto aprendizaje del lenguaje R a lo largo de este trabajo, ya que aunque los dos miembros que conforman el equipo ten\'ian 
conocimientos previos de dicho lenguaje, se ha realizado un estudio mucho m\'as profundo en \'el.
Otro punto a destacar es la gran cantidad de campos en los que se pueden aplicar los algoritmos desarrollados aqu\'i, ya que se podr\'an 
utilizar para extraer conocimiento en redes sociales, con fines m\'edicos, para obtener datos a partir de los art\'iculos que se compran 
en una tienda, etc. Luego, esperamos que todo este trabajo, pueda servir para ayudar y facilitar el trabajo a los investigadores o a cualquier persona 
que est\'e interesada en este campo.

La versi\'on final que se ha obtenido en este TFG, en realidad, es el principio de un trabajo que seguir\'a realizando el equipo del 
departamento de Matem\'atica aplicada de la UMA, ya que se espera mejorar las funciones actuales, as\'i como a\~nadir otras nuevas 
que ayuden a completar el paquete para el An\'alisis Formal de Conceptos. As\'i que, a partir de este, se podr\'an iniciar nuevos TFGs u otro 
tipo de trabajos de investigaci\'on, el equipo del director del proyecto en el que destacamos a Pablo Cordero y Manuel Enciso, seguir\'an 
con el desarrollo del paquete, tanto directamente como promoviendo nuevos TFG. Incluso otros grupos de investigaci\'on nos han pedido 
hacer uso del paquete para usarlo en sistemas de recomendaci\'on tur\'istica.

Por \'ultimo, y como conclusi\'on personal para este trabajo, me gustar\'ia destacar el gran aprendizaje de trabajo en grupo, ya que al ser 
un proyecto realizado en com\'un con un compa\~nero, las decisiones hab\'ia que tomarlas conjuntamente, lo que ha resultado bastante f\'acil y 
en todo momento he encontrado en mi compa\~nero a alguien dispuesto a ayudar en cualquier peque\~no detalle o en situaciones en las que 
necesitaba una mano amiga. Ha sido un placer trabajar codo con codo con \'el. Tampoco puedo olvidar en este espacio a nuestro tutor, \'Angel Mora, 
un profesional con paciencia infinita al ayudarnos a comprender toda la base matem\'atica relacionada con este proyecto, as\'i 
como una gran persona a la que se puede acudir con cualquier tema que se necesite. Gracias Jes\'us. Gracias \'Angel.
