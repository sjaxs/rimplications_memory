\section{Conclusiones y trabajos futuros}

A lo largo de este TFG se ha desarrollado una librer\'ia que se utilizar\'a para el An\'alisis Formal de Conceptos. Este es un 
campo muy amplio, por lo que se ha requerido de un estudio amplio tanto del lenguaje R, como del campo de AFC y la base matem\'atica en la que 
est\'a fundamentado. A partir de este estudio, se procedi\'o a la implementaci\'on de los algoritmos que se han expuesto en ambas partes de 
este trabajo, as\'i como la creaci\'on de alguno de ellos, o la adaptaci\'on de los que el departamento ya hab\'ia comenzado.

Se quiere destacar el alto aprendizaje del lenguaje R a lo largo de este trabajo, ya que aunque los dos miembros que conforman el equipo ten\'ian 
conocimientos previos de dicho lenguaje, se ha realizado un estudio mucho m\'as profundo en \'el.
Otro punto a destacar es la gran cantidad de campos a los que se pueden aplicar los algoritmos desarrollados aqui, ya que se podr\'an 
utilizar 

La versi\'on final que se ha obtenido en este TFG, en realidad, es el principio de un trabajo que seguir\'a realizando el equipo del 
departamento de Matem\'atica aplicada de la UMA, ya que se espera mejorar las funciones actuales, as\'i como an\~nadir otras nuevas 
que ayuden a completar el paquete para el An\'alisis Formal de Conceptos.