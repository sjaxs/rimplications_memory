\subsection{Librer\'ia de An\'alis de Conceptos Formales}

\textbf{Introducci\'on}

El ana\'alis de conceptos formales es un entorno conceptual que estructura, analiza y visualiza conocimiento a partir de datos proporcionados 
como una relaci\'on binaria entre objetos y atributos.

Como se ha explicado anteriormente, un contexto formal \( (G, M, I) \) consiste en dos conjuntos, G y M, y una relaci\'on binaria \( I \subseteq G x M \). Los elementos 
de G se llaman objetos, los de M, atributos de \( (G, M, I) \). Si \(g \in G ~ y ~ m \in M \) son una relaci\'on I, se escribir\'a 
\( (g,m) \in I ~ o ~ g I m \) y se leer\'a como ``el objeto g tiene el atributo m".

Para trabajar con contextos se utilizan varias funciones que se van a desgranar una a una en los siguientes puntos:

    \subsubsection{Leer concepto}

        \textbf{Descripci\'on}

        Como primera funci\'on se tiene la de obtener un concepto formal con su estructura adecuada teniendo la informaci\'on en un fichero. A la 
        hora de utilizar esta funci\'on se pueden usar ficheros xls, txt o r, ya que el resultado ser\'a el mismo, una variable de tipo data.frame 
        con toda la informaci\'on del fichero de entrada.

        \textbf{C\'odigo}

        \lstinputlisting{r_code/FCA/read.fc.R}

        \textbf{Ejemplo}



    %Mp y mp2
    \subsubsection{Operador de derivaci\'on de atributos a objetos}

        \textbf{Descripci\'on}

        Dado un contexto formal \( (G, M, I) \), se llama operador de derivaci\'on de atributos a objetos a la 
        funci\'on \( ^\uparrow : 2^G \rightarrow 2^M \)

        \( ^\uparrow = { m \in M | <g,m> \in I ~ para ~ todo ~ g \in A } \)

        El conjunto \( A^\uparrow \) es el de los atributos compartidos por todos los objetos de A, es decir, este operador devolver\'a 
        un conjunto de attributos que ser\'an los que contiene un determinado objeto.


        \textbf{C\'odigo}

        \lstinputlisting{r_code/FCA/mp.R}


        \textbf{Ejemplo}


    \subsubsection{Operador de doble derivaci\'on de atributos a objetos}

        \textbf{Descripci\'on}

        El conjunto que devuelve el operador \( A^\uparrow \) tambi\'en se suele representar como \( A' \), por lo que en este caso, 
        el operador de doble derivaci\'on de atributos a objetos se representar\'a como \( A'' \).


        \( A'' = (A^\uparrow)^\downarrow \)

        Es decir, al conjunto que se obtiene como salida del operador de derivaci\'on de atributos a objetos, se le aplica el operador 
        de derivaci\'on de objetos a atributos.


        \textbf{C\'odigo}

        \lstinputlisting{r_code/FCA/mp2.R}


        \textbf{Ejemplo}


    %Gp y Gp2
    \subsubsection{Operador de derivaci\'on de objetos a atributos}

        \textbf{Descripci\'on}

        Dado un contexto formal \( (G, M, I) \), se llama operador de derivaci\'on de objetos a atributos a la 
        funci\'on \( ^\downarrow : 2^M \rightarrow 2^G \)

        \( ^\downarrow = { g \in G | <g,m> \in I ~ para ~ todo ~ m \in B } \)

        El conjunto \( B^\downarrow \) es el de los objetos compartidos por todos los atributos de B, en este caso, este operador devolver\'a 
        un conjunto de objetos que ser\'an los que contiene un determinado atributo.

        \textbf{C\'odigo}

        \lstinputlisting{r_code/FCA/gp.R}

        \textbf{Ejemplo}


    \subsubsection{Operador de doble derivaci\'on de objetos a atributos}

        \textbf{Descripci\'on}

        El conjunto que devuelve el operador \( B^\downarrow \) tambi\'en se suele representar como \( B' \), por lo que en este caso, 
        el operador de doble derivaci\'on de objetos a atributos se representar\'a como \( B'' \).


        \( B'' = (B^\downarrow)^\uparrow \)

        Es decir, al conjunto que se obtiene como salida del operador de derivaci\'on de objetos a atributos, se le aplica el operador 
        de derivaci\'on de atributos a objetos.


        \textbf{C\'odigo}

        \lstinputlisting{r_code/FCA/gp2.R}


        \textbf{Ejemplo}



    \subsubsection{Crear contexto}

        \textbf{Descripci\'on}

        En esta librer\'ia, un contexto consta de una lista con dos vectores diferentes. Luego, con esta funci\'on, se puede crear un contexto 
        pas\'andole como par\'ametros dos vectores diferentes.
        Cada uno de dichos vectores, tendr\'a como nombre g y m en este orden, y representar\'an los objetos y atributos.


        \textbf{C\'odigo}

        \lstinputlisting{r_code/FCA/create.context.R}

        \textbf{Ejemplo}



    \subsubsection{Concepto formal}

    
        \textbf{Descripci\'on}

        Dado un contexto formal \( K = (G, M, I) \), donde \( A \subseteq G ~ y ~ B \subseteq M \). Se dice que un par \( <A,B> \) es un 
        concepto formal de \(K\) si \( A^\uparrow = B ~ y ~ B^\downarrow = A\).
        
        Es decir, \( <A,B> \) ser\'a un concepto formal se A contiene exactamente todos los objetos que compoarten los atributos de B, y a su 
        vez, B contiene exactamente todos los atributos que comparten los objetos de A.

        \textbf{C\'odigo}

        \lstinputlisting{r_code/FCA/is.formal.concept.R}

        \textbf{Ejemplo}



    %infimum.fc y infimum.set.fc
    \subsubsection{\'Infimo}

    
        \textbf{Descripci\'on}

        Siendo \((A_{1}, B_{1}) ~ y ~ (A_{2}, B_{2})\) dos conceptos formales de un contexto formal, se define el \'infimo como:

        El superconcepto com\'un m\'as peque\~no de \((A_{1}, B_{1}) ~ y ~ (A_{2}, B_{2})\):

        \( (A_{1}, B_{1}) \vee (A_{2}, B_{2}) = ((A_{1} \cup A_{2})'', ~ B_{1}\cap B_{2}) \)

        \textbf{C\'odigo}

        \lstinputlisting{r_code/FCA/infimum.R}

        \lstinputlisting{r_code/FCA/infimum.set.fc.R}

        \textbf{Ejemplo}



    %supremum.fc y supremum.set.fc
    \subsubsection{Supremo}

    
        \textbf{Descripci\'on}

        Siendo \((A_{1}, B_{1}) ~ y ~ (A_{2}, B_{2})\) dos conceptos formales de un contexto formal, se define el supremo como:

        El subconcepto com\'un m\'as grande de \((A_{1}, B_{1}) ~ y ~ (A_{2}, B_{2})\):

        \( (A_{1}, B_{1}) \wedge (A_{2}, B_{2}) = (A_{1} \cap A_{2}'', ~ (B_{1}\cup B_{2})'') \)


        \textbf{C\'odigo}

        \lstinputlisting{r_code/FCA/supremum.R}

        \lstinputlisting{r_code/FCA/supremum.set.fc.R}


        \textbf{Ejemplo}



    \subsubsection{...object.concept.fc}

    
        \textbf{Descripci\'on}

        Si se tiene un contexto y un objeto, se pueden extraer los atributos de dicho objeto en el contexto. Esto es precisamente lo que hace 
        esta funci\'on, ya que crea un nuevo contexto a partir de uno inicial y un objeto con ayuda de los operadores de derivaci\'on.


        \textbf{C\'odigo}

        \lstinputlisting{r_code/FCA/object.concept.fc.R}


        \textbf{Ejemplo}



    \subsubsection{Objetos de un concepto}

    
        \textbf{Descripci\'on}

        Esta funci\'on devolver\'a todos los objetos de un concepto...


        \textbf{C\'odigo}

        \lstinputlisting{r_code/FCA/all.object.concept.R}


        \textbf{Ejemplo}



    \subsubsection{...attribute.concept.fc}

        En este caso, si se tiene un contexto y un atributo, se pueden extraer los objetos de dicho atributo en el contexto. 
        Esto es lo que hace esta funci\'on, crea un nuevo contexto a partir de uno inicial y un atributo con ayuda de los operadores 
        de derivaci\'on.
    
        \textbf{Descripci\'on}


        \textbf{C\'odigo}

        \lstinputlisting{r_code/FCA/attribute.concept.fc.R}


        \textbf{Ejemplo}



    \subsubsection{Atributos de un concepto}

    
        \textbf{Descripci\'on}

        Esta funci\'on devolver\'a todos los atributos de un concepto...

        \textbf{C\'odigo}

        \lstinputlisting{r_code/FCA/all.attribute.concept.fc.R}

        \textbf{Ejemplo}



    \subsubsection{... all.fc}

    
        \textbf{Descripci\'on}


        \textbf{C\'odigo}

        \lstinputlisting{r_code/FCA/all.fc.R}


        \textbf{Ejemplo}



\subsection{Minimal generator}


    \textbf{Introducci\'on}


    \textbf{Descripci\'on}


    \textbf{C\'odigo}


    \textbf{Ejemplo}

