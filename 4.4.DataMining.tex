\subsection{Librer\'ia de An\'alis de Conceptos Formales}

\textbf{Introducci\'on}

El ana\'alis de conceptos formales es un entorno conceptual que estructura, analiza y visualiza conocimiento a partir de datos proporcionados 
como una relaci\'on binaria entre objetos y atributos.

Como se ha explicado anteriormente, un contexto formal \( (G, M, I) \) consiste en dos conjuntos, G y M, y una relaci\'on binaria \( I \subseteq G x M \). Los elementos 
de G se llaman objetos, los de M, atributos de \( (G, M, I) \). Si \(g \in G ~ y ~ m \in M \) son una relaci\'on I, se escribir\'a 
\( (g,m) \in I ~ o ~ g I m \) y se leer\'a como ``el objeto g tiene el atributo m".

Para trabajar con contextos se utilizan varias funciones que se van a desgranar una a una en los siguientes puntos:

    \subsubsection{Leer concepto}

        \textbf{Descripci\'on}

        Como primera funci\'on se tiene la de obtener un concepto formal con su estructura adecuada teniendo la informaci\'on en un fichero. A la 
        hora de utilizar esta funci\'on se pueden usar ficheros xls, txt o r, ya que el resultado ser\'a el mismo, una variable de tipo data.frame 
        con toda la informaci\'on del fichero de entrada.

        \textbf{C\'odigo}


        \textbf{Ejemplo}



    %Mp y mp2
    \subsubsection{Operador de derivaci\'on de atributos a objetos}

        \textbf{Descripci\'on}

        Dado un contexto formal \( (G, M, I) \), se llama operador de derivaci\'on a la funci\'on \( ^\uparrow : 2^G \rightarrow 2^M \)

        \( ^\uparrow = { m \in M | <g,m> \in I ~ para ~ todo ~ g \in A } \)

        El conjunto es el de los atributos compartidos por todos los objetos de A.

        \textbf{C\'odigo}


        \textbf{Ejemplo}

    %Gp y Gp2
    \subsubsection{Operador de derivaci\'on de objetos a atributos}

        \textbf{Descripci\'on}


        \textbf{C\'odigo}


        \textbf{Ejemplo}

    \subsubsection{Crear contexto}

        \textbf{Descripci\'on}


        \textbf{C\'odigo}


        \textbf{Ejemplo}

    \subsubsection{Concepto formal}

    
        \textbf{Descripci\'on}


        \textbf{C\'odigo}


        \textbf{Ejemplo}

    \subsubsection{.... G iguales}

        
        \textbf{Descripci\'on}


        \textbf{C\'odigo}


        \textbf{Ejemplo}


    %infimum.fc y infimum.set.fc
    \subsubsection{\'Infimo}

    
        \textbf{Descripci\'on}


        \textbf{C\'odigo}


        \textbf{Ejemplo}


    %supremum.fc y supremum.set.fc
    \subsubsection{Supremo}

    
        \textbf{Descripci\'on}


        \textbf{C\'odigo}


        \textbf{Ejemplo}


    \subsubsection{...object.concept.fc}

    
        \textbf{Descripci\'on}


        \textbf{C\'odigo}


        \textbf{Ejemplo}


    \subsubsection{Objetos de un concepto}

    
        \textbf{Descripci\'on}


        \textbf{C\'odigo}


        \textbf{Ejemplo}


    \subsubsection{...attribute.concept.fc}

    
        \textbf{Descripci\'on}


        \textbf{C\'odigo}


        \textbf{Ejemplo}


    \subsubsection{Atributos de un concepto}

    
        \textbf{Descripci\'on}


        \textbf{C\'odigo}


        \textbf{Ejemplo}


    \subsubsection{... all.fc}

    
        \textbf{Descripci\'on}


        \textbf{C\'odigo}


        \textbf{Ejemplo}



\subsection{Minimal generator}


    \textbf{Introducci\'on}


    \textbf{Descripci\'on}


    \textbf{C\'odigo}


    \textbf{Ejemplo}