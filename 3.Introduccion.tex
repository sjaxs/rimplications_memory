\section{Introducci\'on}

En el presente TFG se va a realizar un paquete de funciones en 
R para el \'area de An\'alisis Formal de Conceptos que se ha 
constituido como una herramienta formal para el an\'alisis de datos, 
la extracci\'on de conocimiento a partir de un 
conjunto de objetos y las propiedades que cumplen dichos objetos.

El director del TFG (\'Angel Mora) est\'a desarrollando junto 
con su grupo de la Universidad de M\'alaga (Pablo Cordero y Manuel 
Enciso), su investigaci\'on en el \'area descrita. Actualmente, 
tienen como gran objetivo el desarrollo de una librer\'ia para el 
lenguaje R (R package) que implemente los algoritmos m\'as 
importantes que han desarrollado para la manipulaci\'on del 
conocimiento en An\'alisis Formal de Conceptos. Este TFG 
constituye la primera versi\'on del paquete que pretendemos sea 
referente para la manipulaci\'on de conocimiento en aplicaciones 
pr\'acticas. Como segundo objetivo, se pretende que este paquete, tras
ser publicado, act\'ue como mecanismo de difusi\'on de los trabajos 
publicados por el grupo de investigaci\'on. En la actualidad, aumentar el 
n\'umero de citas de las publicaciones es prioritario.

Por \'ultimo, AFC es probablemente la herramienta m\'as desconocida en las \'areas 
de ciencia de datos y de descubrimiento de conocimiento. Se pretende que, puesto que en los 
ambientes de data mining, machine learning, big data, etc. el lenguaje R es uno de los m\'as 
usados, nuestro paquete y su uso de \'el, impulse el conocimiento del \'area de AFC.

Se trata de un trabajo fin de grado realizado en la modalidad en equipo. 
Dicho equipo est\'a formado por Jes\'us Alberto Dom\'inguez \'Alvarez y por m\'i, 
Ana Esperanza Villal\'on Mart\'in. El trabajo se ha realizado de forma conjunta en su 
gran mayor\'ia, aunque ha habido partes desarrolladas de forma individual por cada uno 
de los miembros del equipo.

La parte com\'un del trabajo ha consistido en el estudio avanzado de R, estudio de c\'omo 
realizar y distribuir un paquete en R para el repositorio CRAN, manual de usuario y documentaci\'on del 
paquete completo.

La parte individual de este TFG presentada en esta memoria ha consistido en el desarrollo de una librer\'ia b\'asica 
para la implementaci\'on de todos los algoritmos que se presentar\'an en ambas partes de este TFG. Otro punto importante es 
el desarrollo de dos funciones de generaci\'on aleatoria de sistemas de implicaciones y contextos. Y por \'ultimo, una librer\'ia 
con algoritmos para el An\'alisis Formal de Conceptos. Todos estos apartados se ir\'an desarrollando en los diferentes puntos de esta memoria.
\\
\\

\textbf{Motivaci\'on:}

En la actualidad, no hay una herramienta que permita el an\'alisis 
optimizado de sistemas de implicaciones en R, y aunque existen otras 
librer\'ias en otros lenguajes, no son r\'apidos o los resultados que 
muestran no llegan a ser efectivos.
Por ello, en este TFG se quiere crear un paquete de funciones que pueda 
ser difundido y usado a nivel general por cualquier persona que se dedique 
al an\'alisis de datos.
\\
\\


\textbf{Objetivos:}

El objetivo principal de este trabajo es obtener una librer\'ia que 
nos permita el an\'alisis r\'apido y efectivo de un conjunto de datos 
y la extracci\'on de conocimiento mediante el An\'alisis Formal de Conceptos y sistemas de implicaciones.

Como segundo objetivo, se encuentra el de distribuir este paquete en el repositorio 
de R, para que cualquier persona interesada, desarrolladores, o investigadores de este \'ambito o cualquiera 
de las \'areas anteriormente citadas, puedan acceder a \'el de forma sencilla.
\\
\\



\thispagestyle{empty}
\mbox{}
