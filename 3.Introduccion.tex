\section{Introducci\'on}

En el presente TFG se va a realizar un paquete de funciones en 
R para el \'area de An\'alisis de Conceptos formales que se ha 
constituido como una herramienta formal para el an\'alisis de datos, 
que permitir\'a la extracci\'on de conocimiento a partir de un 
conjunto de objetos y las propiedades que cumplen dichos objetos.
\\
El director del TFG (\'Angel Mora) est\'a desarrollando junto 
con su grupo de la Universidad de M\'alaga (Pablo Cordero y Manuel 
Enciso), su investigaci\'on en el \'area descrita. Actualmente, 
tienen como gran objetivo el desarrollo de una librer\'ia para el 
lenguaje R (R package) que implemente los algoritmos m\'as 
importantes que han desarrollado para la manipulaci\'on del 
conocimiento en An\'alisis de Conceptos Formales. Este TFG 
constituye la primera versi\'on del paquete que pretendemos sea 
referente para la manipulaci\'on de conocimiento en aplicaciones 
pr\'acticas.
\\
\\
Motivaci\'on:
En la actulidad, no hay una herramienta que permita el an\'alisis 
optimizado de sistemas de implicaciones en R, y aunque existen otras 
librer\'ias en otros lenguajes, no son r\'apidos o los resultados que 
muestran no llegan a ser efectivos.
Por ello, en este TFG se quiere crear un paquete de funciones que pueda 
ser difuncido y usado a nivel general por cualquier persona que se dedique 
al an\'alisis de datos.
\\
\\
Objetivos:
El objetivo principal de este trabajo es obtener una librer\'ia que 
nos permita el an\'alisis r\'apido y efectivo de un conjunto de datos 
y obtener las conclusiones correctas mediante el An\'alisis de Conceptos 
Formales y sistemas de implicaciones.

Como segundo objetivo, se encuentra el de distribuir este paquete en el repositorio 
de R, para que cualquier persona interesada pueda acceder a \'el de forma sencilla.
\\
\\


\newpage
\thispagestyle{empty}
\mbox{}

\newpage